\documentclass{report}
\usepackage{url,amssymb}%general
\usepackage[show]{ed} %local
\usepackage{moreverb}
\def\omdoc{OMDoc}
\title{Translating into a New Mathematical Language\\
  A Feasibility Study}
\author{\begin{minipage}{5cm}Michael Kohlhase\\
  Carnegie Mellon University\\
  \url{kohlhase+@cs.cmu.edu}
\end{minipage}\qquad
\begin{minipage}{5cm}Fairouz Kamareddine\\
  Heriot Watt University\\
  \url{fairouz@cee.hw.ac.uk}
\end{minipage}}
\begin{document}
\maketitle
\begin{abstract}
We go from CML to NML\ednote{complete}
\end{abstract}
\newpage
\tableofcontents
\newpage
\section{Introduction}

bla bla\ednote{copy from the NML paper or the proposal}

\section{The Pipeline}

In this section we will work and an extended example that was also discussed
in~\cite{paper}; In a mathematical textbook we might find the following fragment.

\begin{quote}
{\bf Definition.} Let $h \not = 0$, let $f$ be a function from $A$ to
$\mathbb{R}$, $a \in A$ and $a+h \in A$.  Then $\frac{f(a+h) -
f(a)}{h}$ is the {\em difference quotient\/} of $f$ in $a$ with
difference $h$. We call $f$ {\em differentiable\/} at $x=a$ if
$\lim_{h \rightarrow 0}\frac{f(a+h) - f(a)}{h}$ exists.
The function $\sqrt{|x|}$ is not differentiable at 0. 
\end{quote} 

\begin{small}
\begin{boxedverbatim}
{\bf Definition.} Let $h \not = 0$, let $f$ be a function from $A$ to
$\mathbb{R}$, $a \in A$ and $a+h \in A$.  Then $\frac{f(a+h) -
f(a)}{h}$ is the {\em difference quotient\/} of $f$ in $a$ with
difference $h$. We call $f$ {\em differentiable\/} at $x=a$ if
$\lim_{h \rightarrow 0}\frac{f(a+h) - f(a)}{h}$ exists.
The function $\sqrt{|x|}$ is not differentiable at 0. 
\end{boxedverbatim}
\end{small}
\subsection{The {\omdoc} Original}
\ednote{discuss the {\omdoc} here, say something about {\omdoc} in general}

\subsection{New Mathematical Lanugage (NML)}
\vspace{-0.08in}
Both definitions require a context.  For convenience we first abbreviate:

$\Gamma_1 = A \subseteq \mathbb{R}, f : A \rightarrow \mathbb{R}, a : A,
h : \mathbb{R}, h \not = 0, a+h \in A $
and 
$\Gamma_2 = A \subseteq \mathbb{R}, f : A \rightarrow \mathbb{R}, a : A$ 

The book consists of the
$\circ$-concatenation of the following three lines:

\vspace{-0.08in}
\begin{quote} 
\begin{tabular}{l c l} 
$\Gamma_1$ & $~~\triangleright ~~$ & $ {\it the~difference~quotient~of~}f
~ := ~ \frac{f(a+h)-f(a)}{h}$ \\ $\Gamma_2$ & $~~\triangleright ~~$ & $ f
{\rm ~is~differentiable~at~}a := \lim_{h \rightarrow
0}\frac{f(a+h)-f(a)}{h} {\rm ~exists~}$ \\ $\emptyset$ & $ ~~
\triangleright ~~$ & $ \neg (\lambda_{x : \mathbb{R}}(\sqrt{|x|}) {\rm
~is~differentiable~at~}0)$
\end{tabular}
\end{quote} 
\vspace{-0.08in}
A derivation leading to the book given above, needs many
small steps, but can be constructed straightforwardly. We omit the
derivation itself. We only give a number of remarks regarding it:
\vspace{-0.08in}
\begin{itemize} 
\item In the  book we type $A$ by declaring it to be a
{\it subset\/} of $\mathbb{R}$. This is not according to the
rules. However, we may consider $A \subseteq \mathbb{R}$ to be
shorthand for $x \in {\cal P}(\mathbb{R})$, which is a proper
declaration. (The power set operator ${\cal P}$ should be accounted
for in the preface, with weak in-type $(\tiset)$ and weak out-type $\tiset$.)
\item Equality, addition, subtraction and division should also be
weakly typed in the preface. 
The same holds for $\raisebox{.5ex}{$\sqrt{}$}$, $|\ldots |$ and the 
logical operator $\neg$.
\item {\it Function application\/} as in $f(a)$ and $f(a+h)$ can be
treated as a binary constant ${\tt appl}$ with weak in-type $(\ttt \times
\ttt)$ and out-type $\ttt$.
\end{itemize} 

\begin{appendix}
\chapter{The Documents in the Pipeline}
\section{The Original}
  {\scriptsize\listinginput[5]{1}{original.tex}}
\section{The corresponding {\omdoc}}
{\scriptsize\listinginput[5]{1}{../../examples/feasibility/orig.omdoc}}
\end{appendix}
\bibliographystyle{alpha}
\ednotemessage
\end{document}


%%% Local Variables:
%%% mode: latex
%%% TeX-master: t
%%% End:
