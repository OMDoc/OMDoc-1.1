\chapter{Conclusion}\label{sec:concl}\label{sec:future}

With {\omdoc} we have proposed a content-based markup format that
allows to represent mathematical knowledge at various levels. As a
consequence the format allows to capture the semantics and structure
of various kinds of mathematical documents, including articles,
textbooks, interactive books, and courses. 

We have argued that the problem of representing mathematical knowledge
has to be addressed at three levels, corresponding to the three levels
of structure found in documents.
\begin{description}
\item[The formula level] is concerned with representing mathematical objects as
  mathematical/logical formulae. {\omdoc} leverages the existing {\openmath} and
  {\mathml} standards for this.
\item[The statement level] consists of statements about the mathematical objects,
  like definitions, theorems and proofs. On this level, {\omdoc} supplies original
  markup schemes that allow structured representations of the mathematical content
  (including both formal and informal elements of representation).
\item[The theory level] allows to group statements to conceptual units according
  to the assumptions on the mathematical objects they describe. An inheritance
  mechanism allows to specify the acessibility and scoping of symbols, and re-use
  flexibly parts of specifications. The theory level even allows to structure
  collections of theories by theory-inclusions and transport theories and proof
  methods along these relations.
\end{description}
We have motivated and described version 1.1 of the {\omdoc} language
and presented an {\xml} document type definition for it. We have
surveyed a set of transformation tools that generate
presentation-oriented documents for human consumption and
machine-oriented documents for communication with mathematical
software systems.

We have developed first {\indextoo{authoring tool}s} for {\omdoc} that
try to simplify generating {\omdoc} documents for the working
mathematician. There is a simple {\omdoc} mode for {\ttin{emacs}}, and
a {\LaTeX} style~\cite{Kohlhase:corfl00} that can be used to generate
{\omdoc} representations from {\LaTeX} sources and thus help migrate
existing mathematical documents. A second step will be to integrate
the {\LaTeX} to {\openmath} conversion tools. 

The next steps in the development will be to develop {\omdoc} version
2.0 including more disruptive changes to the language, including a
re-organization of central {\omdoc} elements like
{\element{definition}}.


%%% Local Variables: 
%%% mode: latex
%%% TeX-master: "omdoc"
%%% End: 
