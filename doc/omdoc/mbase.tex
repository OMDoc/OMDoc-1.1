\ifpdf\section{MBase, an Open Mathematical Knowledge Base}
\else\section{{\mbase}, an Open Mathematical Knowledge Base}\fi\label{sec:mbase}
\begin{center}\large\sf Andreas Franke and Michael Kohlhase\\
\url{http://www.mathweb.org/mbase}
\end{center}

We describe the {\mbase} system, a web-based mathematical knowledge base
(see~\url{http://www.mathweb.org/mbase}). It offers the infrastructure for a
universal, distributed repository of formalized mathematics. Since it is
independent of a particular deduction system and particular logic, the {\mbase}
system can be seen as an attempt to revive the {\sc Qed} initiative from an
infrastructure viewpoint. See~\cite{KohFra:rkcimss00} for the logical issues
related to supporting multiple logical languages while keeping a consistent
overall semantics. The system is realized as a mathematical service in the
{\mathweb} system~\cite{FraKoh:mabdl99}, an agent-based implementation of a
mathematical software bus for distributed mathematical computation and knowledge
sharing. The content language of {\mbase} is {\omdoc}.

We will start with a description of the system from the implementation point of
view (we have described the data model and logical issues
in~\cite{KohFra:rkcimss00}).  

The {\mbase} system is realized as a distributed set of {\mbase} servers (see
figure~\ref{fig:mbase}). Each {\mbase} server consists of a Relational Data Base
Management System (RDBMS) connected to a {\mozart} process
(yielding a {\mathweb} service) via a standard data base interface.
For browsing the {\mbase} content, any {\mbase} server provides an {\tt
  http} server (see {\url{http://mbase.mathweb.org:8000}} for an example) that
dynamically generates presentations based on {\html} or {\xml} forms.

This architecture combines the storage facilities of the RDBMS with the
flexibility of the concurrent, logic-based programming language
{\oz}~\cite{Smolka:Oz:95}, of which {\mozart} (see
{\url{http://www.mozart-oz.org}}) is a distributed implementation .  Most
importantly for {\mbase}, {\mozart} offers a mechanism called {\defin{pickling}},
which allows for a limited form of persistence: {\mozart} objects can be
efficiently transformed into a so-called pickled form, which is a binary
representation of the (possibly cyclic) data structure.  This can be stored in a
byte-string and efficiently read by the {\mozart} application effectively
restoring the object.  This feature makes it possible to represent complex objects
(e.g. logical formulae) as {\oz} data structures, manipulate them in the {\mozart}
engine, but at the same time store them as strings in the RDBMS. Moreover, the
availability of ``{\oz}lets'' ({\mozart} functors) gives {\mbase} great
flexibility, since the functionality of {\mbase} can be enhanced at run-time by
loading remote functors. For instance complex data base queries can be compiled by
a specialized {\mbase} client, sent (via the Internet) to the {\mbase} server and
applied to the local data e.g. for specialized searching
(see~\cite{Duchier:tntb98} for a related system and the origin of this idea).

\begin{figure}
  \begin{center}
    \ifpdf\includegraphics[width=10cm]{mbase.pdf}
    \else\includegraphics[width=10cm]{mbase.eps}\fi
    \caption{System Architecture}
    \label{fig:mbase}
  \end{center}
\end{figure}

{\mbase} supports transparent distribution of data among several {\mbase} servers
(see~\cite{KohFra:rkcimss00} for details). In particular, an object $O$ residing
on an {\mbase} server $S$ can refer to (or depend on) an object $O'$ residing on a
server $S'$; a query to $O$ that needs information about $O'$ will be delegated to
a suitable query to the server $S'$.  We distinguish two kinds of {\mbase} servers
depending on the data they contain: {\em archive servers\/} contain data that is
referred to by other {\mbase}s, and {\em scratch-pad\/} {\mbase}s that are not
referred to. To facilitate caching protocols, {\mbase} forces archive servers to
be {\em conservative}, i.e. only such changes to the data are allowed, that the
induced change on the corresponding logical theory is a conservative extension.
This requirement is not a grave restriction: in this model errors are corrected by
creating new theories (with similar presentations) shadowing the erroneous ones.
Note that this restriction does not apply to the non-logical data, such as
presentation or description information, or to scratchpad {\mbase}s making them
ideal repositories for private development of mathematical theories, which can be
submitted and moved to archive {\mbase}s once they have stabilized.

%%% Local Variables: 
%%% mode: latex
%%% TeX-master: "omdoc"
%%% End: 
