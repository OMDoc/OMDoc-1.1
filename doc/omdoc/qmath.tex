\ifpdf\section{QMath: An Authoring Tool for OMDoc}
\else\section{{\qmath}: An Authoring Tool for {\omdoc}}\fi\label{sec:qmath}
\begin{center}\large\sf Alberto Gonz\'alez Palomo\\
\url{http://www.matracas.org}
\end{center}

{\qmath} is a batch processor that produces an {\omdoc} file from a plain Unicode
text document. The purpose of {\qmath} is to allow fast writing of mathematical
documents, using plain text and a straightforward syntax (like in computer algebra
systems) for mathematical expressions .

The ``Q'' was intended to mean ``quick'', since {\qmath} began in 1998 as an
abbreviated notation for {\mathml}. The first version (0.1) just expanded the
abbreviations to full {\mathml} element names, and added the extra markup such as
{\tt{<mrow>}} and the like.  There have been many changes (and two complete
rewrites) since then. You can find a more detailed history at
{\url{http://www.matracas.org/qmath/history.html}}

{\qmath} is very simple in its design: it just parses a text (UTF-8) file
according to a user-definable table of symbols, and builds an XML document from
that. The symbol definitions are grouped in files called ``contexts''. The idea is
that when you declare a context, its file is loaded and from then on these symbol
definitions take precedence over any previous one, thus setting the context for
parsing of subsequent expressions.

The text is split into ``paragraphs'', which are pieces of text separated by at
least one empty line. Each paragraph can have a metadata section at the beginning.
There are a variety of classes of paragraphs, which are identified by a name
followed by a colon (``:''), optionally followed by an identifier which becomes
the {\attribute{id}{*}} attribute of the generated {\omdoc} element. The text is
put in a {\tt{<CMP>}} inside a container element which depends on the paragraph
type. This can be anything allowed by {\omdoc}, such as {\tt{<assertion>}} ,
{\tt{<axiom>}}, or the default {\tt{<omtext>}} if the paragraph doesn't have a
{\qmath} paragraph type label. Inside the text, a mathematical expression is
enclosed in dollar (``{\tt{\$}}'') signs.  Each such a section becomes an
{\element{OMOBJ}} element in the output document.

\begin{myfig}{qmathex}{A minimal {\qmath} document and its {\omdoc} result}\scriptsize
\begin{boxedverbatim}
QMATH 0.3.6
:"Diary"
:Winston Smith
:1984-04-04
:en

Context: "Mathematics/Arithmetic"
Context: "Mathematics/OMDoc"

Theory:[<-thoughtcrime]

:"Down with Big Brother"
Freedom is the freedom to say $2+2=4$. 
If that is granted, all else follows.
\end{boxedverbatim}
\quad
\begin{tabular}[b]{|p{5.2cm}|}\hline
From {\url{contexts/en/Mathematics/OpenMath/arith1.qmath}}:\\
{\tt{Symbol: plus OP\_PLUS "arith1:plus"}}\\
{\tt{Symbol: + OP\_PLUS "arith1:plus"}}\\
{\tt{Symbol: sum APPLICATION "arith1:sum"}}\\
{\tt{Symbol: \( \Sigma \) APPLICATION "arith1:sum"}}\\
 $\cdots$\\\hline

From {\url{contexts/en/Mathematics/OpenMath/relation1.qmath}}:\\
{\tt{Symbol: = OP\_EQ "relation1:eq"}}\\
{\tt{Symbol: neq OP\_EQ "relation1:neq"}}\\
{\tt{Symbol: \ensuremath{�}= OP\_EQ "relation1:neq"}}\\
{\tt{Symbol: \( \neq \) OP\_EQ "relation1:neq"}}\\
 $\cdots$\\\hline
\end{tabular}
\\\footnotesize
\begin{boxedverbatim}
<?xml version='1.0' encoding='UTF-8' standalone='no'?>
<!DOCTYPE omdoc PUBLIC "-//OMDoc//DTD OMDoc V1.1//EN" 
                       "http://www.mathweb.org/omdoc/omdoc.dtd" []>
<omdoc lang='en'>
 <metadata lang='en'>
  <Title xmlns='http://purl.org/DC'>Diary</Title>
  <Contributor xmlns='http://purl.org/DC' role='aut'>
   Winston Smith
  </Contributor>
  <Date xmlns='http://purl.org/DC'>1984-04-04</Date>
 </metadata>
 <theory id='thoughtcrime'>
  <omtext>
   <metadata>
    <Title xmlns='http://purl.org/DC'>Down with Big Brother</Title>
   </metadata>
   <CMP>
    Freedom is the freedom to say 
    <OMOBJ xmlns='http://www.openmath.org/OpenMath'>
     <OMA>
      <OMS cd='relation1' name='eq'/>
      <OMA><OMS cd='arith1' name='plus'/><OMI>2</OMI><OMI>2</OMI></OMA> 
      <OMI>4</OMI>
     </OMA>
    </OMOBJ>.
    If that is granted, all else follows. 
   </CMP>
  </omtext>
 </theory>
</omdoc>
\end{boxedverbatim}
\end{myfig}

{\Myfigref{qmathex}} shows a minimal {\qmath} document, and the {\omdoc} document
generated from it.  The first line ("{\tt{QMATH 0.3.6}}") in the {\qmath} document
is required for the parser to recognize the file. The lines beginning with
``{\tt{:}}'' are metadata items: first the document title, then the author name
(one line for each author), and finally the primary language for the document.
This last item is required, as it sets the basic symbol set accordingly. For
example, the ``{\tt{Context}}'' item of an English document is written
``{\tt{Contexto}}'' if the document is in Spanish.  (Similarly, the arithmetic
context would be "{\tt{Matem\'aticas/Aritm\'etica}}") The document is split into
paragraphs, which are separated by empty lines.  Then, mathematical expressions
are written enclosed by ``{\tt{\$}}'' (dollar) signs.

The {\tt{QMath}} command works as a pure filter: reads the document from standard
input, and writes the resulting {\omdoc} in standard output. So, the typical usage
is
\begin{verbatim}
QMath <document.qmath > document.omdoc
\end{verbatim}
It needs the {\tt{QMATH\_HOME}} environment variable to contain the path for the
root {\qmath} directory, where it can find the "contexts" directory. For example,
if you have the {\tt{contexts}} directory at {\url{/tmp/qmath_3/contexts}}, you
should set {\tt{QMATH\_HOME}} to {\url{/tmp/qmath_3}}

{\qmath} is distributed under the GNU General Public License (GPL):
{\url{http://www.gnu.org/licenses/licenses.html#GPL}}

%%% Local Variables: 
%%% mode: latex
%%% TeX-master: "omdoc"
%%% End: 

