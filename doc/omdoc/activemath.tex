\section{Project {\ifpdf{ActiveMath}\else{\activemath}\fi}}\label{sec:activemath}
\begin{center}\large\sf
Erica Melis, Eric Andr\`es, Jochen B\"udenbender, Adrian Frischauf,\\
     George Goguadze, Paul Libbrecht, Martin Pollet, Carsten Ullrich\\
\url{http://www.activemath.org/}
\end{center}

In a nutshell, {\activemath} is a generic web-based learning system that
dynamically generates interactive (mathematics) courses adapted to the student's
goals, preferences, capabilities, and knowledge.  The content is represented in
{\omdoc} format with several extensions needed in an educational context. For each
user, the appropriate content is retrieved from the knowledge base {\mbase} and
the course is generated individually according to pedagogical rules.  Then the
course is presented to the user via a standard web-browser. One of the exceptional
features of {\activemath} is its integration of stand-alone mathematical service
systems.

Currently, a minimal authoring kit and a translation tool from restricted {\LaTeX}
to {\omdoc} are provided to support authoring {\omdoc}s\footnote{See
  \url{http://www.activemath.org/~paul/AuthoringComments/} for a description.}.

In the near future, the authoring tools will have intelligent features and
{\activemath} will integrate a user-adaptive suggestion and feedback mechanism in
addition to the adaptive course generation. A comprehensive description of the
system can be found in~\cite{activemathAIEDJ01}.

\subsection{OMDoc Extensions}
The {\activemath} DTD is an extension of the general {\omdoc} DTD, in particular,
with pedagogically motivated extensions such as difficulty or abstractness of an
example or exercise.

{\activemath} will also differentiate exercises according to their type with
values defining the user's required activity such as {\it check-question}, {\it
  make-hypothesis}, {\it prove}, {\it model} or {\it explore} where e.g.,
\textit{explore} means interactive exploration with the help of specified external
system (such as \texttt{maple}, {\OMEGA}, or statistics software).

Additional pedagogically motivated metadata elements will be introduced such as
{\tt field} (e.g., {\it computer science},{\it math}, {\it economy}) and
{\tt{learning\-context}} with values corresponding to school and university
levels\footnote{These will be used to provide different content to users with
  different background fields and at different levels.} in accordance with the
Learning Object Metadata Standard\footnote{\url{http://ltsc.ieee.org/wg12/}}.
Furthermore, the author can specify the pedagogical goal of an exercise, example,
or elaboration, that is whether learning this item increases {\it knowledge}, {\it
  comprehension}, {\it application}, or {\it transfer}.

Finally, in {\activemath}, relations are going to be represented by the element
{\tt relation}, defined in the {\activemath} DTD and classified by introducing a
type of the relation. The type values are: {\it depends-on}, {\it
  counterexam\-ple-for}, {\it similar-example}, {\it similar-exercise}, {\it
  citation}, etc. The need for distinguishing the types of relations arises not
only in educational contexts.

Since we need certain additional {\omdoc} elements in the educational context, a
common representation for proof methods, proof plans, algorithms will be added in
the future, hopefully some of them even to the common {\omdoc} itself.

%??verbosity ?? highlight


\subsection{Adaptive Presentation}

{\activemath} offers dynamically constructed courses that suit the learner's
learning goals, her choosen learning scenarios, her presentation preferences, and
knowledge mastery. To realize this, {\activemath} maintains a user model and its
presentation tools include a course generator and pedagogical rules employed by
the course generator.

Presentation of the content is currently made in {\html} through {\xslt}
transformations with adaptation to the users' taste through CSS filters.
{\omdoc}'s semantic encoding allows to envision other output formats and some of
them are under work.

\subsection{Integration of External Systems}

Currently, {\activemath} integrates the Computer Algebra Systems MuPad and Maple
and the proof planner of {\OMEGA}, and statistics software. Moreover, an external
student/exercise management systems will be integrated in 2002.  The distributed
web-architecture of {\activemath} is well-suited for integrating external systems
and also the OMDoc representation is -- in principle -- a basis for integrating
different systems.

Currently however, exercises and examples cannot simply pass {\omdoc}s or
{\openmath} elements to the mathematical service systems because
{\openmath}-phrasebooks are not available for most systems.  An instruction on how
to write the OMDocs for exercises for which an external system is called can be
found in {\url{http://www.ags.uni-sb.de/~adrianf/activemath}}. The abstract
description of an exercise includes startup, shutdown, and eval instructions.

\subsection{Current Status}

The {\activemath} learning environment is alpha status of development.  Most of
the basic features are becoming stable and new ones are being planned.  Authoring
tools are under development but usage of QMath (see other implementations) is
recommended and compatible.

More information and a demo version of {\activemath} can be found from our
web-page {\url{http://www.activemath.org/}}.
%%% Local Variables: 
%%% mode: latex
%%% TeX-master: "omdoc"
%%% End: 
