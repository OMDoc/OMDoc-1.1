\chapter{Introduction}\label{sec:intro}

It is plausible to expect that the way we do (i.e. conceive, develop, communicate
about, and publish) mathematics will change considerably in the next
nine\footnote{In the release document of {\omdoc}1.0~\cite{Kohlhase:otormd00} we
  claimed that it would change in the next 10, and that is one year ago.} years.
The Internet plays an ever-increasing role in our everyday life, and most of the
mathematical activities will be supported by mathematical software systems (we
will call them {\emin{mathematical services}}\index{service!mathematical})
connected by a commonly accepted distribution architecture, which we will call the
{\emin{mathematical software bus}\index{software bus}\index{bus!software}}. We
will subsume all proposed architectures and implementations of this
idea~\cite{FraHes:aoidms99,FraKoh:mabdl99,DenCol:tpt00,ArmZin:timrs00} by the term
{\mathweb}\index{mathweb@{\mathweb}}. We believe that interoperability based
on communication protocols will eventually make the constructions of bridges
between the particular implementations simple, so that the combined systems appear
to the user as one homogeneous web.

One of the tasks that have to be solved is to define an open markup language for
the mathematical objects and knowledge exchanged between mathematical services.
The {\omdoc} format presented in this report attempts to do this by providing an
infrastructure for the communication and storage of {\emin{mathematical
    knowledge}}\index{knowledge!mathematical}.

In chapter~\ref{chap:markup} we will describe the status quo of mathematical
markup schemes before {\omdoc} and show that these markup schemes -- while giving
a good basis -- are not sufficient for content-based markup of mathematical
knowledge. They do not provide markup for mathematical forms like
{\indextoo{definition}s}, {\indextoo{theorem}s}, and {\indextoo{proof}s} that have long
been considered paradigmatic of mathematical documents like textbooks and papers.
They also leave implicit the large-scale structure of mathematical knowledge. In
particular, it has traditionally been structured into mathematical
theories\index{mathematical theory}\index{theory} that serve as a situating
{\indextoo{context}} for all forms of mathematical communication.

In chapter~\ref{chap:omdoc}, we define the {\omdoc} markup primitives and motivate
them from either particular structures in mathematical documents or from
processing needs of computer-supported mathematics. As all mathematical
communication is in the form of (or can be transcribed to) {\indextoo{mathematical
    document}}s\index{document!mathematical} such as {\indextoo{publication}}s,
{\indextoo{overhead slide}}s, {\indextoo{letter}}s, {\indextoo{e-mail}}s, in/output from
mathematical software systems, {\omdoc} uses documents as a guiding intuition for
mathematical knowledge with the goal of providing a framework, where all of these
forms can be accommodated. In accordance with this motivation {\omdoc} provides a
rich mix of elements of informal and formal mathematics. To model particular kinds
of documents in {\omdoc} usually only a subset will be needed, e.g. informal ones
for traditional mathematical textbooks, or formal ones for communication of
software systems. However, availability of both kinds of markup primitives in
{\omdoc} allow to develop novel kinds of mathematical documents, where formal and
informal elements are intimately intermixed.

We will discuss current and intended applications of the {\omdoc} format in
chapter~\ref{chap:processing} and discuss which applications will need which parts
of the {\omdoc} format. 

Finally, the appendix contains useful materials like the {\omdoc} document type
definition, and a quick reference table. 

\section*{{\omdoc} Version 1.1}

This document describes version 1.1 of the {\omdoc} format. Version 1.0 has been
released on November 1. 2001, after about 18 Months of development, to give
developers a stable interface to base their systems on. It has been adopted by
various projects in automated deduction, algebraic specification and
computer-supported education. The experience from these projects has uncovered a
multitude of small deficiencies and extension possibilities of the format, that
have been discussed in the {\omdoc} community. Version 1.1 is an attempt to roll
the uncontroversial and non-disruptive part of the extensions and corrections into
a consistent language format. We have tried to keep the changes to version 1.0
conservative, adding optional attributes or child elements.

In some cases we had to introduce non-conservative changes, to repair design flaws
and inconsistencies of version 1.0. One example is the {\element{hpothesis}}
element that has received a required attribute
{\attribute{discharged-in}{hypothesis}} that is necessary for specifying the scope
of local assumptions in proofs, and cannot be inferred from the context. To
minimize disruption we have tried to keep changes like this one to a minimum for
the elements that are in frequent use today.  We are working on a new version
({\omdoc}2.0) that will incorporate re-organizations of central features of
{\omdoc} like the {\element{definition}} element.

We have however re-organized some parts of the {\omdoc} format that are currently
less used in the anticipation that this will make them more effective. Examples
are the representations of complex theories (see sections~\ref{sec:morphisms}
to~\ref{sec:parametric-theories}) or the organization of non-{\xml} data
(section~\ref{sec:private}).

Finally, we have added new features that were missing from {\omdoc}1.0 and turned out
to be important for the enterprise of representing mathematical knowledge.
Examples of this are a new referencing scheme for {\omdoc} elements in
section~\ref{sec:catalogue} and a new way of specifying presentation for {\omdoc}
elements. In both cases, the method that was used in {\omdoc}1.0 for symbols is
extended and generalized to arbitrary {\omdoc} elements. These extensions have
found their way into {\omdoc}1.1, even though they are not totally fixed yet,
since we anticipate to gain implementation experience for {\omdoc}2.0. They are
non-disruptive, since they are strictly additional.

An element-by-element account of the changes is tabulated in appendix
{\ref{sec:changelog}}.

\section*{Acknowledgments}
 
Of course the {\omdoc} format has not been developed by one person alone, the
original proposal was taken up by several research groups, most notably the
{\OMEGA} group at Saarland University, the {\inka} and {\activemath} projects at
the German Research Center of Artificial Intelligence (DFKI), the RIACA group at
the Technical University of Eindhoven, the {\sc In2Math} project at the University
of Koblenz, and the {\sc CourseCapsules} project at Carnegie Mellon University.
They have discussed the initial proposals, represented their materials in {\omdoc}
and in the process refined the format with numerous suggestions and discussions
(see {\url{http://www.mathweb.org/~mailists/omdoc}} for the archive of the
{\omdoc} mailing list.)

The author specifically would like to thank Serge Autexier, Olga Caprotti, David
Carlisle, Claudio Sacerdoti Coen, Arjeh Cohen, Armin Fiedler, Andreas Franke,
George Goguadze, Dieter Hutter, Erica Melis, Paul Libbrecht, Martijn Oostdijk,
Alberto Palomo Gonzales, Martin Pollet, Julian Richardson, Manfred Riem, and
Michel Vollebregt for their input, discussions and feedback from implementations
and applications.

The work presented in this report was supported by the ``Deutsche
Forschungsgemeinschaft'' in the special research action ``Resource-adaptive
cognitive processes'' (SFB 378), and a five-year Heisenberg Stipend to the author.
Carnegie Mellon University and SRI International have supported the author while
working on revisions for version 1.1.


%%% Local Variables: 
%%% mode: latex
%%% TeX-master: "omdoc"
%%% End: 
