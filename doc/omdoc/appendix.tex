\begin{appendix}
\chapter{Errata to the released Specification}
Here we track the errata in the {\omdoc} 1.1 specification (see
{\url{http://www.mathweb.org/omdoc/archive/omdoc1.1.{ps,pdf}}}). These errata have
been cleared in this version.
\section*{1 Introduction}

No errata known.
\section*{2 Mathematical Markup Schemes}
No errata known.
\section*{3 OMDoc Elements}
No errata known.
\subsection*{3.1 Metadata for Mathematical Elements}
No errata known.

\subsection*{3.2 Mathematical Statements}

\subsubsection*{3.2.1 Specifying Mathematical Properties}
\begin{itemize}
\item The text fails to make clear for the
  {\attribute{id}{om:*}}/{\attribute{xref}{om:*}} to {\openmath} objects are only
  allowed, if the referencing element has the same name as the referenced one. In
  particular, there are no implicit conversions.
  
\item The text fails to mention the {\attribute{logic}{FMP}} attribute of the
  {\element{FMP}} element. We need to add ``{\element{FMP}}s always appear in
  groups, which can differ in the value of their {\attribute{logic}{FMP}}
  attribute, which specifies the logical formalism. The value of this attribute
  specifies the logical system used in formalizing the content. All members of the
  {\indextoo{multi-logic {\tt{FMP}} group}}\index{FMP group!multi-logic} have to
  formalize the same mathematical object or property, i.e. they have to be
  translations of each other.''
\end{itemize}

\subsubsection*{3.2.2 Symbols, Definitions, and Axioms}
\begin{enumerate}
\item The values {\attval{obj}{type}{definition}} and
  {\attval{simple}{type}{definition}} were overlapping, and the role of the
  {\element{FMP}} and {\element{OMOBJ}} children of the {\element{definition}} was
  unclear. The value {\attval{obj}{type}{definition}} has been dropped and we have
  clarified that in simple definitions, the {\element{OMOBJ}} is the substitution
  element whereas the {\element{FMP}} captures the meaning of the {\element{CMP}}
  group in Logic.
\item The attribute {\attribute{kind}{symbol}} of the {\element{symbol}} element
  can also have the value {\attval{sort}{kind}{symbol}} for sets that are
  inductively built up from constructor symbols
\end{enumerate}

\subsubsection*{3.2.3 Assertions and Alternatives}
No errata known.
\subsubsection*{3.2.4 Mathematical Examples in {\ifpdf{OMDoc}\else{\omdoc}\fi}}
No errata known.
\subsubsection*{3.2.5 Representing Proofs in {\ifpdf{OMDoc}\else{\omdoc}\fi}}
No errata known.
\subsubsection*{3.2.6 Abstract Data Types}
The optional {\element{recognizer}} element should be a child of
{\element{sortdef}}, and not of the {\element{constructor}} element. The
specification text and examples are correct, but the quick reference table is
incorrect.

\subsection*{3.3 Theories as Mathematical Contexts}
\subsubsection*{3.3.1 Simple Inheritance}
The content model for {\element{theory}} in Figure 3.22 is incorrect. It should
read {\tt{commonname*, CMP*, ({\rm statement} | inclusion, imports)*}}. Moreover,
the attribute model has a spurious comma. Furthermore, the text should make clear
that {\omdoc}1.1 does not allow theories to nest and that theories can include
{\element{imports}} statements.

\subsubsection*{3.3.2 Inheritance via Translations}
No errata known.

\subsubsection*{3.3.3 Statements about Theories}
The text does not make this clear, but the elements {\element{theory-inclusion}},
{\element{axiom-inclusion}} and {\element{decomposition}} may not occur in a
{\element{theory}} element. Worse, the document type definition allows this as
well. 

\subsubsection*{3.3.4 Parametric theories in {\ifpdf{OMDoc}\else{\omdoc}\fi}}
No errata known.

\subsection*{3.4 Auxiliary Elements}

\subsubsection*{3.4.1 Preservation of Text Structure}
\begin{description}
\item[Figure 3.29:] Specifying Tables with {\tt{<omgroup type="dataset">}} The first
label for the second axis is ``{\tt{b11}}''. Should be ``{\tt{b1}}''.
\item[omgroup] The DTD did not contain value {\attval{narrative}{type}{omgroup}} that was present in
  the specification.
\end{description}

\subsection*{3.4.4 Exercises}
In the {\element{solution}} element, where {\element{proof}} was allowed, we have
to allow {\element{proofobject}} as well. 

\subsection*{3.5 Adding Presentation Information to \ifpdf{OMDoc}\else{\omdoc}\fi}
\begin{enumerate}
\item It should be made clear that the {\attribute{xml:lang}{use,xslt,style}} attribute of the
  {\element{use}}, {\element{xslt}} and {\element{style}} elements does not have the default
  value {\ttin{en}}.
\item {\omdoc}1.1 uses the {\attribute{style}{*}} attribute for all elements that
  have an {\attribute{id}{*}} attribute to specify generic style classes for the
  {\omdoc} elements.  This is based on a misunderstanding of the {\xml}
  {\indextoo{cascading style sheet}} ({\css}) mechanism~\cite{BosHak:css98}, which
  uses the {\attribute{class}{*}} attribute to specify this information and uses the
  {\attribute{style}{*}} attribute to specify {\css} directives that override the class
  information. 
  
  {\sf Even though this is a grave error (it severely limits the usefulness) we
    will not change it in the {\omdoc} 1.1 specification and wait for the release
    of {\omdoc} 1.2 to fix this, the renaming of the {\attribute{style}{*}}
    attribute to {\attribute{class}{*}} would break existing implementations.}
\end{enumerate}

\subsubsection{3.5.2 Specifying the Notation of Mathematical Symbols}
\begin{description}
\item[Figure 3.44] the example still uses the {\omdoc}1.0 version of specifying
  {\xslt} content via the {\attribute{system}{use}} attribute, in {\omdoc}1.1 the
  element {\element{xslt}} should be used. 
\end{description}


\subsection*{3.6 Identifying and Referencing {\ifpdf{OMDoc}\else{\omdoc}\fi} Elements}

\subsection*{Locating {\ifpdf{OMS}\else{\element{OMS}}\fi} elements by the {\ifpdf{OMDoc}\else{\omdoc}\fi} Catalogue}
No errata known.

\subsection*{A URI-based Mechanism for Element Reference}
The text does not make it clear that the namespace prefixes for theory collections
can be declared in any element that dominates the referencing element. The DTD
does not allow this either. We will not change this in the DTD, since the changes
are too disruptive and {\omdoc}1.2 is coming up soon. 

\subsection*{Uniqueness Constraints and Relative URI references}
No errata known.
\section*{4 OMDoc Applications, Tools,  and Projects}
No errata known.

\section*{B Changes}
  No errata known.
\section*{C Quick-Reference for OMDoc Elements}
No errata known.
\section*{D Quick-Reference for OMDoc Attributes}
No errata known.
\section*{E OMDoc DTD}
We use the following {\indextoo{public identifier}} for DTDs: {\ttin{-//OMDoc//DTD OMDoc V1.1//EN}}
\begin{enumerate}
\item The {\attribute{xml:lang}{use, xslt, style}} attribute of the {\element{use}},
  {\element{xslt}} and {\element{style}} elements should not have the default value
  {\ttin{en}}.
\item The elements {\element{theory-inclusion}}, {\element{axiom-inclusion}} and
  {\element{decomposition}} may not occur in a {\element{theory}} element.
\item The content model for {\element{solution}} should make the {\element{FMP}},
  {\element{proof}}, and {\element{proofobjects}} elements optional.
\item the attribute {\attribute{for}{exercise}} should have been optional
\item the optional {\element{recognizer}} element should be a child of {\element{sortdef}},
  and not of the {\element{constructor}} element.
\item the {\element{exercise}} element should allow multiple mathematical objects,
  not at most one.
\item the {\ttin{\%cfm;}} parameter entity in the DTD did not allow for multiple
  FMPs, even though the spefication says that they appear in multi-logic groups.
\item the default {\attribute{precedence}{presentation}} attribute of the
  {\element{presentation}} element should have the default value 1000. 
\item the content model of the {\element{definition}} element had to be adapted to
  the clarification in the specification. The old version only allowed
  {\element{CMP}}-only content with {\tt{rxp}}. The value
  {\attval{obj}{type}{definition}} has also been dropped.
\item the content model for the {\element{omdoc}} element prescribed at least one
  omdoc item. This is not intended, since we want to allow catalogue-only
  documents for administrative purposes. 
\item the theory attribute for the {\element{assertion}} {\element{alernative}},
  {\element{proof}} and {\element{proofobject}} elements should be of type
  {\tt{CDATA}}, after all it contains a URI that points to an external theory.
\item the specification mentions type {\attval{comment}{type}{omtext}}, but the
  DTD did not allow this.
\end{enumerate}

%%% Local Variables: 
%%% mode: latex
%%% TeX-master: "errata"
%%% End: 



\chapter{Changes from Version 1.0}\label{sec:changelog}

In this section we will keep a log on the changes that have occurred in the
released versions of the {\omdoc} format. We will briefly tabulate the changes by
element name. For the state of an element we will use the shorthands ``dep'' for
deprecated (i.e. the element is no longer in use in the new {\omdoc} version),
``cha'' for changed, if the element is re-structured (i.e.  some additions and
losses), ``new'' if did not exist in the old {\omdoc} version, and finally ``aug''
for augmented, i.e. if it has obtained additional children or attributes in the
new {\omdoc} version.

Version 1.1 is mainly a bug-fix release that has become necessary by the
experiments of encoding legacy material in {\omdoc}. The changes are relatively
minor, mostly added optional fields. The only non-conservative changes concern the
{\element{private}}, {\element{hypothesis}}, {\element{sortdef}} and
{\element{signature}} elements.  {\omdoc} files can be upgraded to version 1.1
with the {\xslt} style sheet
{\url{http://www.mathweb.org/omdoc/xsl/omdoc1.0adapt1.1.xsl}}.

\begin{center}\footnotesize
\begin{longtable}{|l|c|p{7.1cm}|l|}\hline
  element                   & state       & comments & cf.\\\hline\hline
{\element{attribute}} & new
     & presentation of attributes for {\xml} elements 
     & \pageref{eldef:attribute}\\\hline
{\element{alternative}} & cha
     & new form of the {\oldelement{alternative-def}{1.1}} element, it can now also used
       as an alternative to {\element{axiom}}. Compared to
       {\oldelement{alternative-def}{1.1}} it has a new optional 
       attribute {\attribute{generated-by}{alternative}} to show that an
       assertion is generated by expanding a some other element like {\element{adt}}.
     & \pageref{eldef:alternative}\\\hline
{\oldelement{alternative-def}{1.1}} & dep
     & new form is {\element{alternative}}, since there can be alternative
     {\element{axiom}s} too.
     & \\\hline
{\element{argument}}    & cha
     & attribute {\attribute{sort}{argument}} is now of type IDREF, since it must be local
     in the definition.
     & \pageref{eldef:argument}\\\hline
{\element{assertion}}      & aug 
     & more values for the {\attribute{type}{assertion}}, new optional attribute
     {\attribute{generated-by}{assertion}} to show that an assertion is generated by
     expanding a {\element{definition}} or an {\element{adt}}. New optional
     attribute {\attribute{proofs}{assertion}}.
     & \pageref{eldef:assertion}\\\hline
{\oldelement{assertion-just}{1.1}} & dep
     & this is now {\element{obligation}} & \\\hline 
{\element{axiom}}          & aug 
     & new  optional attribute {\attribute{generated-by}{axiom}} to show that an
       axiom is generated by expanding a {\element{definition}}.
     & \pageref{eldef:axiom}\\\hline
{\element{axiom-inclusion}} & cha
     & now allows a {\element{CMP}} group for descriptive text, 
       includes a set of {\element{obligation}s} instead of an
       {\oldelement{assertion-just}{1.1}}.  The {\tt{timestamp}}
       attribute is deprecated, use {\element{dc:Date}} with appropriate
       {\attribute{action}{Date}} instead
     & \pageref{eldef:axiom-inclusion} \\\hline
{\element{CMP}}           & cha
     & the attribute {\tt{format}} is now deprecated, it makes no
     sense, since we are more strict and consistent about {\element{CMP}} content.
     & \pageref{eldef:CMP}\\\hline
{\element{code}}        & cha 
     & Attributes {\attribute{width}{omlet}} and {\attribute{height}{omlet}} now in
       {\element{omlet}}, got  attributes {\attribute{classid}{code}} and
       {\attribute{codebase}{code}} from  {\element{private}}. Attribute
       {\attribute{format}{data}} moved to {\element{data}} children.  

       The multilingual group of {\element{CMP}} elements for description is
       deprecated, use {\element{metadata}}/{\element{Description}} instead.      
       Child element {\element{data}} may appear multiple times (with 
       different values of the {\attribute{format}{data}}).
     & \pageref{eldef:code}\\\hline
{\element{constructor}}    & aug 
     & new optional child {\element{recognizer}} for a recognizer predicate
     & \pageref{eldef:constructor}\\\hline
{\oldelement{Coverage}{1.1}} & dep
     &  this Dublin Core element specifies the place or time which the
     publication's contents addresses. This does not seem appropriate for the
     mathematical content of {\omdoc}. 
     & \\\hline
{\element{data}}           & aug
     & new optional attributes {\attribute{size}{data}} to specify the size of the data
       file that is referenced by the {\attribute{href}{data}} attribute and
       {\attribute{format}{data}} for the format the data is in.
     & \pageref{eldef:data}\\\hline
{\tt{dc:*}}                 & aug 
     & {\element{Contributor}}, {\element{Creator}}, {\element{Publisher}} have 
       received an optional {\attribute{id}{dc:*}} 
       attribute, so that they can be cross-referenced by the new 
       {\attribute{who}{Date}} of the {\element{Date}} element.
     & \pageref{eldef:Title}\\\hline
{\tt{dc:{\element{Date}}}} & aug 
     & new optional {\attribute{who}{Date}} attribute that can be used to specify who
     did the {\attribute{action}{Date}} on this date.
     & \pageref{eldef:Date}\\\hline
{\tt{dc:{\oldelement{Translator}{1.1}}}} & dep
     & this element is not part of Dublin Core, it got into {\omdoc} by mistake, we
     use {\element{Contributor}} with {\attribute{role}{dc:*}}={\tt{trl}} for this.
     & \pageref{eldef:Contributor}\\\hline
{\element{decomposition}} & aug
     & has a new required {\attribute{id}{decomposition}} attribute. 
       It is no longer a child of {\element{theory-inclusion}}, but specifies which
       {\element{theory-inclusion}} it justifies by the new required attribute
       {\attribute{for}{decomposition}}. 
     & \pageref{eldef:decomposition} \\\hline
{\element{definition}}      & aug
     & new optional children {\element{measure}} and {\element{ordering}} 
       to specify termination of recursive definitions.  New optional 
       attribute {\attribute{generated-by}{definition}}  to show that it 
       is generated by expanding a {\element{definition}}.
     &\pageref{eldef:definition} \\\hline
{\element{element}} & new
     & presentation of {\xml} elements 
     & \pageref{eldef:element}\\\hline
{\element{FMP}}     & aug
     & now allows multiple {\element{conclusion}} elements, to represent general
       Gentzen-type sequents (not only natural deduction.)
     & \pageref{eldef:FMP}\\\hline
{\element{hypothesis}}     & cha
     & new {\bf required} attribute {\attribute{discharged-in}{hypothesis}} to specify the 
       {\element{derive}} or {\element{conclude}} element that discharges 
       this hypothesis. 
     & \pageref{eldef:hypothesis}\\\hline
{\element{measure}}         & new
     & specifies a measure function (as an OMOBJ)
     & \pageref{eldef:measure}\\\hline
{\element{metadata}}        & aug   
     & new optional attribute {\attribute{inherits}{metadata}} that allows to inherit 
       metadata from other declarations 
     & \pageref{eldef:metadata}\\\hline
{\element{method}} & cha
     & first child that used to be an {\element{OMSTR}} or {\element{ref}} element
     is now moved into a required {\attribute{xref}{method}} attribute that holds an URI
     that points to the element that defines the method. The {\element{OMOBJ}}
     content of the other children (they were {\element{paramter}} elements) is
     now directly included in the {\element{method}} element.
     &\pageref{eldef:method}\\\hline
{\element{obligation}} & new
     & takes over the role of {\oldelement{assertion-just}{1.1}}.
     & \\\hline
{\element{omgroup}}        & aug
     & also allows the elements that can only appear in {\element{theory}}
       elements, so that {\element{omgroup}}s can also be used for grouping inside
       {\element{theory}} elements. The
       {\attribute{type}{omgroup}} attribute is now restrained to one of 
       {\attval{narrative}{type}{omgroup}}, {\attval{sequence}{type}{omgroup}}, 
       {\attval{alternative}{type}{omgroup}}, {\attval{contrast}{type}{omgroup}}.
     & \pageref{eldef:omgroup}\\\hline
{\element{omlet}}          & aug
     & obtained attributes {\attribute{width}{omlet}} and {\attribute{height}{omlet}} from
       {\element{private}}. New optional attributes {\attribute{action}{omlet}}  
       for the action to be taken when 
       activated, and {\attribute{data}{omlet}} a URIref to data in a private
       element. New optional attribute {\attribute{type}{omlet}} for the type of
       the applet.     
     & \pageref{eldef:omlet}\\\hline
{\element{omstyle}} & new
     & for specifying the style of {\omdoc} elements
     & \pageref{eldef:omstyle}\\\hline
{\element{omtext}} & cha
     & the {\attribute{from}{omtext}} is deprecated, we only leave the {\attribute{for}{omtext}}
       attribute, to specify the referential character of the {\attribute{type}{omtext}}.
     &\pageref{eldef:omtext}\\\hline
{\element{ordering}}         & new
     & specifies a well-founded ordering (as an OMOBJ)
     & \pageref{eldef:ordering}\\\hline
{\oldelement{parameter}{1.1}}      & dep
     & the {\element{OMOBJ}} element child is now directly a child of
     {\element{method}}
     & \\\hline
{\element{pattern}}         & cha
     & the child can be an arbitrary{\openmath} element.
     & \pageref{eldef:pattern}\\\hline
{\element{premise}}         & cha
     & new optional attribute {\attribute{rank}{premise}} for the importance in
       the inference rule. The old {\tt{href}} attribute is renamed to  
       {\attribute{xref}{premise}} to be consistent with other cross-referening.
     & \\\hline
{\element{presentation}}    & aug
     &  {\attribute{id}{presentation}} attribute is now optional. new attribute
        {\attribute{xref}{presentation}} that allows to inherit the information
        from another {\element{presentation}} element. New attribute
        {\attribute{theory}{presentation}} to specify the theory the symbol is from; without
        this, referencing in {\omdoc} is not unique. 
     & \pageref{eldef:presentation}\\\hline
{\element{private}}        & cha
     & new optional attribute {\attribute{for}{private}} to point to an {\omdoc} element it 
       provides data for. As a consequence, {\element{private}} elements 
       are no longer allowed in other {\omdoc} elements, only on top-level. 
       New attribute {\attribute{replaces}{private}} as a pointer to the {\omdoc} 
       elements that are replaced by the system-specific information in 
       this element. Old attributes {\attribute{width}{omlet}} and
       {\attribute{height}{omlet}} now in {\element{omlet}}. Attribute
       {\attribute{format}{data}} moved to {\element{data}} children.

       The multilingual group of {\element{CMP}} elements for description is
       deprecated, use {\element{metadata}}/{\element{Description}} instead.
       
       Child element {\element{data}} may appear multiple times (with 
       different values of the {\attribute{format}{data}}). The
       attributes {\attribute{classid}{private}} and
       {\attribute{codebase}{private}} are deprecated, since they only make sense on the
       {\element{code}} element.
     & \pageref{eldef:private}\\\hline
{\element{proof},\element{proofobject}} & cha
     & attribute {\attribute{theory}{proof}} is now optional, since it can appear
       in a {\element{theory}}.
     & \pageref{eldef:proof}\\\hline
{\element{recognizer}}   & new
     & specifies the recognizer predicate of a sort.
     & \pageref{eldef:recognizer}\\\hline
{\element{recurse}} & new
     &  recursive calls to presentation in {\element{style}}.
     & \pageref{eldef:recurse}\\\hline
{\element{ref}}            & cha
     & attribute {\attribute{kind}{ref}} renamed to {\attribute{type}{kind}}.
     & \pageref{eldef:ref}\\\hline
{\element{selector}}     & cha
     & the old {\attribute{type}{selector}} attribute (had values 
     {\tt{total}} and {\tt{partial}}) is deprecated, its duty is now carried by an attribute
     {\attribute{total}{selector}} (values {\attval{yes}{total}{selector}} and 
     {\attval{no}{total}{selector}}). 
     & \pageref{eldef:selector}\\\hline
{\oldelement{signature}{1.1}}      & dep & for the moment & \\\hline
{\element{sortdef}}         & cha
     & attribute {\attribute{id}{sortdef}} is now mandatory, otherwise the defined
     symbol  no name. The {\tt{kind}} that was fixed to {\tt{sort}} is
     deprecated, this piece of information is redundant.
     &  \pageref{eldef:sortdef} \\\hline
{\element{style}} & new
     & allows to specify style information in {\element{presentation}} and
       {\element{omstyle}} elements using a simplified {\omdoc}-internalized
       version of {\xslt}.
     & \pageref{eldef:style}\\\hline
{\element{symbol}}          &  aug
     & new optional attribute {\attribute{generated-by}{symbol}}  to show that it
        is generated by expanding a {\element{definition}}.
     & \pageref{eldef:symbol}\\\hline
{\element{text}} & new
     & presentation of text in {\element{omstyle}}.
     & \pageref{eldef:text}\\\hline
{\element{theory-inclusion}} & cha
     & now allows {\element{CMP}} group  for descriptive text, 
       no longer has a {\element{decomposition}} child, this is now attatched  by
       its {\attribute{for}{decomposition}} attribute. The {\tt{timestamp}}
       attribute is deprecated, use {\element{dc:Date}} with appropriate
       {\attribute{action}{Date}} instead.
     & \pageref{eldef:theory-inclusion}\\\hline
{\element{type}} & aug
     & can now also appear on top-level. Has an optional {\attribute{id}{type}} attribute
       for identification, and an optional {\attribute{for}{type}} attribute to point to
       a {\element{symbol}} element it declares type information for.
     & \pageref{eldef:type}\\\hline
{\element{use}}            & aug
     & New attribute {\attribute{element}{use}}
       allows to specify that the content should be encased in an XML element with
       the attribute-value pairs specified in the string specified in the attribute
       {\attribute{attributes}{use}}. 
     & \pageref{eldef:use}\\\hline
{\element{value-of}} & new
     & presentation of values in {\element{style}}.
     & \pageref{eldef:value-of}\\\hline
{\element{with}}           & new
     & used to supply fragements of text in {\element{CMP}s} with
       {\attribute{id}{with}} and {\attribute{id}{style}} attributes that can be
       used for presentation and referencing.
     & \pageref{eldef:with}\\\hline
{\element{xslt}}   & new
     & allows to embed {\xslt} into {\element{presentation}} and
       {\element{omstyle}} elements.
     & \pageref{eldef:xslt}\\\hline
\end{longtable}
\end{center}


%%% Local Variables: 
%%% mode: latex
%%% TeX-master: "omdoc"
%%% End: 

\chapter{Quick-Reference Table to the {\ifpdf{OMDoc}\else{\omdoc}\fi} Elements}\label{sec:table}
\def\tabelt#1#2#3#4#5#6{{#1}&\pageref{eldef:#1}&{#2}&{#3}&{#4}&{#5}&{#6}\\\hline}
\def\generalcat{struct}\def\cfmpcat{stat}\def\mathcat{stat}\def\proofcat{prf}\def\constcat{thy}
\def\adtcat{adt}\def\cpxtcat{thy}\def\auxcat{aux}\def\metacat{meta}\def\prescat{pres}\def\excat{ex}
 {\scriptsize
\begin{longtable}{|>{\tt}p{2.1cm}|l|l|>{\tt}p{2cm}|>{\tt}p{2cm}|c|>{\tt}p{3cm}|}\hline
{\rm Element}& p. & Type  & {\rm Required}  & {\rm Optional} & D & Content \\\hline
             & &        & {\rm Attribs}  & {\rm Attribs} & C &        \\\hline\hline
\tabelt{adt}\adtcat{id}{type, style}{+}{CMP*, commonname*, sortdef+}
\tabelt{alternative}\mathcat{id, for, theory, entailed-by, entails,
             entailed-by-thm, entails-thm}{type, generated-by, just-by,
             style}{+}{CMP*, (FMP| requation*| OMOBJ)}
\tabelt{answer}\excat{verdict}{id, style}{+}{{\cfm}}
\tabelt{argument}\adtcat{sort}{}{+}{selector?}
\tabelt{assertion}\mathcat{id}{type, theory, generated-by, style}{+}{{\cfm}}
\tabelt{assumption}\cfmpcat{id}{style}{+}{CMP*, OMOBJ?}
\tabelt{attribute}\prescat{name}{}{--}{(\#PCDATA| value-of| text)*}
\tabelt{axiom}\constcat{id}{generated-by, style}{+}{{\cfm}}
\tabelt{axiom-inclusion}\cpxtcat{id, from, to}{style}{+}{morphism?, (path-just| obligation*)}\hline
\tabelt{choice}\excat{}{id, style}{+}{{\cfm}}
\tabelt{CMP}\cfmpcat{}{xml:lang}{--}{({\rm text}| OMOBJ| with| omlet)*}
\tabelt{code}\auxcat{id, theory}{id, for, theory, pto, pto-version, format,
             requires, type, classid, codebase, width, height, style}{+}{CMP*, input?, output?, effect?, data+}
\tabelt{commonname}\constcat{}{xml:lang}{--}{{\rm{\element{CMP}}content}}
\tabelt{conclude}\proofcat{id}{style}{--}{CMP*, \justmatter}
\tabelt{conclusion}\cfmpcat{id}{style}{+}{CMP*, OMOBJ?}
\tabelt{constructor}\adtcat{id}{type, scope, style}{+}{commonname*, argument*, recognizer?}
\tabelt{Contributor}\metacat{}{id, role, style}{--}{{\tt{\%DCperson}}}
\tabelt{Creator}\metacat{}{id, role, style}{--}{{\tt{\%DCperson}}}\hline
\tabelt{data}\auxcat{}{format, href, size}{--}{<![CDATA[...]]>}
\tabelt{Date}\metacat{}{action, who}{--}{{\indextoo{ISO8601}}}
\tabelt{decomposition}\cpxtcat{links}{}{--}{EMPTY}
\tabelt{definition}\constcat{id, for}{just-by, type, generated-by, style}{+}{CMP*,
             (FMP| requation+| OMOBJ)?, measure?, ordering?}
\tabelt{Description}\metacat{}{xml:lang}{--}{{\rm{\element{CMP}}content}}
\tabelt{derive}\proofcat{id}{style}{--}{CMP*, FMP?, \justmatter}\hline
\tabelt{effect}\auxcat{}{}{--}{CMP*}
\tabelt{element}\prescat{name}{}{--}{(attribute| element| text| recurse)*}
\tabelt{example}\mathcat{id, for}{type, assertion, proof, style}{+}{symbol*, CMP*| OMOBJ?}
\tabelt{exercise}\excat{id}{type, for, from, style}{+}{{\cfm}, hint?, (solution*|mc*)}
\tabelt{extradata}\metacat{}{}{--}{ANY}\hline
\tabelt{FMP}\cfmpcat{}{logic}{--}{(assumption*, conclusion*)|OMOBJ}
\tabelt{Format}\metacat{}{}{--}{{\rm fixed:}"xml, x-omdoc"}\hline
\tabelt{hint}\excat{}{id, style}{+}{{\cfm}}
\tabelt{hypothesis}\proofcat{id, discharged-in, style}{}{--}{{\cf}}\hline
\tabelt{Identifier}\metacat{}{scheme}{--}{ANY}
\tabelt{ignore}\auxcat{}{type, comment}{--}{ANY}
\tabelt{imports}\cpxtcat{id, from}{type, hiding, style}{--}{CMP*, morphism?}
\tabelt{inclusion}\cpxtcat{for}{}{--}{}
\tabelt{input}\auxcat{}{}{--}{CMP*}
\tabelt{insort}\adtcat{for}{}{--}{}\hline
\tabelt{Language}\metacat{}{}{--}{{\indextoo{ISO8601}}}\hline
\tabelt{mc}\excat{}{id, style}{--}{symbol*, choice, hint?, answer}
\tabelt{measure}\constcat{}{}{--}{OMOBJ}
\tabelt{metacomment}\proofcat{}{id, style}{--}{CMP*}
\tabelt{metadata}\metacat{}{inherits}{--}{({\rm dc-element})*, extradata}
\tabelt{method}\proofcat{xref}{}{--}{OMOBJ*}
\tabelt{morphism}\cpxtcat{}{id, base, style}{--}{requation*}\hline
\tabelt{obligation}\cpxtcat{induced-by, assertion}{}{--}{EMPTY}
\tabelt{omdoc}\generalcat{id}{type, version, style, xmlns, catalogue, xmlns:xsi, xsl:schemaLocation}{+}{({\rm {\omdoc} element})*}
\tabelt{omgroup}\generalcat{id}{type, for, from, style}{+}{{\rm {\omdoc}element}*}
\tabelt{omlet}\auxcat{}{id, argstr, type, function, action, data, style}{+}{ANY}
\tabelt{omstyle}\prescat{element}{for, id, xref, style}{--}{(style|xslt)*}
\tabelt{omtext}\generalcat{id}{type, for, from, style}{+}{CMP+, FMP?}
\tabelt{ordering}\constcat{}{}{--}{OMOBJ}
\tabelt{output}\auxcat{}{}{--}{CMP*}\hline
\tabelt{path-just}\cpxtcat{local, globals}{}{--}{EMPTY}
\tabelt{pattern}\constcat{}{}{--}{OMOBJ}
\tabelt{premise}\proofcat{xref}{}{--}{EMPTY}
\tabelt{presentation}\prescat{for}{id, xref, fixity, parent, lbrack, rbrack, separator, 
                        bracket-style, style, precedence, crossref-symbol, theory}
                        {--}{(use | xslt | style)* }
\tabelt{private}\auxcat{}{id, for, theory, pto, pto-version, format, requires,
             type, classid, codebase, width, height, replaces, style}{+}{CMP*, data+}
\tabelt{proof}\proofcat{id, for, theory}{style}{+}{symbol*, CMP*, (metacomment|
             derive| hypothesis)*, conclude}
\tabelt{proofobject}\proofcat{id, for, theory}{style}{+}{CMP*, OMOBJ}
\tabelt{Publisher}\metacat{}{id, style}{--}{ANY}\hline
\tabelt{ref}\generalcat{}{xref, type}{--}{ANY}
\tabelt{recognizer}\adtcat{id}{type, scope, kind, style}{--}{commonname*}
\tabelt{recurse}\prescat{}{select}{--}{EMPTY}
\tabelt{Relation}\metacat{}{}{--}{ANY}
\tabelt{requation}\constcat{}{id, style}{--}{pattern, value}
\tabelt{Rights}\metacat{}{}{--}{ANY}\hline
\tabelt{selector}\adtcat{id}{type, scope, kind, total, style}{--}{commonname*}
\tabelt{solution}\excat{}{id, for, style}{+}{({\cfm}) | proof}
\tabelt{sortdef}\adtcat{id}{kind, scope, style}{--}{commonname*, (constructor|insort)*}
\tabelt{Source}\metacat{}{}{--}{ANY}
\tabelt{style}\prescat{format}{xml:lang, requires}{--}{(element | text | recurse | value-of)*}
\tabelt{Subject}\metacat{}{xml:lang}{--}{{\rm{\element{CMP}}content}}
\tabelt{symbol}\constcat{id}{kind, scope, style}{+}{CMP*, (commonname| type| selector)*}\hline
\tabelt{text}\prescat{}{}{--}{(\#PCDATA)}
\tabelt{theory}\cpxtcat{id}{style}{+}{commonname*, {\rm statement}*}
\tabelt{theory-inclusion}\cpxtcat{id, from, to, by, style}{}{+}{(morphism, decomposition?)}
\tabelt{Title}\metacat{}{xml:lang}{--}{{\rm{\element{CMP}}content}}
\tabelt{type}\constcat{system}{id, for, style}{--}{CMP*, OMOBJ}
\tabelt{Type}\metacat{}{}{--}{{\rm fixed:}"Dataset"{\rm or}"Text"}\hline
\tabelt{use}\prescat{format}{xml:lang, requires, larg-group, rarg-group, fixity,
             lbrack, rbrack, separator, crossref-symbol, element, attributes}
                        {--}{(use | xslt | style)* }
\tabelt{value}\constcat{}{}{--}{OMOBJ}\hline
\tabelt{value-of}\prescat{select}{}{--}{EMPTY}
\tabelt{with}\cfmpcat{id}{style}{--}{{\rm {\element{CMP}} content}}
\tabelt{xslt}\prescat{format}{xml:lang, requires}{--}{CDATA}
\end{longtable}}

\chapter{Quick-Reference Table to the {\omdoc} Attributes}\label{sec:att-table}
\def\atabelt#1#2#3#4{{#1}&{#2}&{#3}\\\hline&\multicolumn{2}{|p{9cm}|}{#4}\\\hline\hline}
{\footnotesize\begin{longtable}{|>{\tt}p{2.5cm}|>{\tt}p{4cm}|>{\tt}p{5cm}|}\hline
{\rm Attribute} & {\em element} & Values \\\hline\hline
\atabelt{action}{omlet}{}
 {specifies the action to be taken when executing the {\tt{omlet}}, the value is
     application-defined.}

\atabelt{argstring}{omlet}{}
 {specifies the argument string for the function specified in the {\tt{function}}
 attribute of  this {\tt{omlet}}}

\atabelt{assertion}{example}{}
 {specifies the assertion that states that the objects given in the example really have
   the expected properties.}

\atabelt{assertion}{obligation}{}
 {specifies the assertion that states that the translation of the statement in the
  source theory specified by the {\tt{induced-by}} attribute is valid in the
  target theory.}

\atabelt{attibutes}{use}{}
 {the attribute string for the start tag of the {\xml} element  substituted for
 the brackets (this is specified in the {\tt{element}} attribute).}
\atabelt{base}{morphism}{}
 {specifies another morphism that should be used as a base for expansion in the
  definition of this morphism}

\atabelt{bracket-style}{presentation, use}{lisp, math  }
 {specifies whether a function application is of the form $f(a,b)$ or $(f a b)$}

\atabelt{catalogue}{omdoc}{}
 {specifies an outside {\omdoc} document that contais catalogue information for
   this one.}

\atabelt{lbrack}{presentation, use}{}
 {the left bracket to use in the notation of a function symbol}

\atabelt{cd}{loc}{}
 {specifies the location of the content dictionary for a theory}

\atabelt{classid, codebase}{code}{}
 {points to a class identifier and codebase, if the {\tt{code}} contains Java.}

\atabelt{comment}{ignore}{}
 {specifies a reason why we want to ignore the contents}

\atabelt{crossref-symbol}{presentation, use}
 {all, brackets, lbrack, no, rbrack, separator, yes}
 {specifies whether crossreferences to
  the symbol definition should be generated in the output format.}

\atabelt{data}{omlet}{}
 {points to a {\tt{private}} element that contains the data for this {\tt{omlet}}}

\atabelt{discharged-in}{hypothesis}{}
 {specifies the scope of a local hypothesis in a proof. It points to the proof
  step which discharges it.}

\atabelt{element}{use}{}
 {the {\xml} element tags to be substituted for the brackets.}

\atabelt{entails, entailed-by}{alternative}{}
 {specifies the equivalent formulations of a definition or axiom}

\atabelt{entails-thm, entailed-by-thm}{alternative}{}
 {specifies the entailsment statements for equivalent formulations of a 
  definition or axiom}

\atabelt{fixity}{presentation}{assoc, infix, postfix, prefix}
  {specifies where the function symbolof a function application should be 
   displayed in the output format}

\atabelt{function}{omlet}{}
 {specifies the function to be called when this {\tt{omlet}} is activated.}

\atabelt{format}{data}{}
 {specifies the format of the data specified by a {\tt{data}} element. The value should
  e.g. be a MIME type.}

\atabelt{generated-by}{symbol, axiom, assertion, definition, alternative}{}
 {points to a higher-level syntax element, that generates this statement.}

\atabelt{for}{*}{}
 {can be used to reference an element by its unique identifier given in its 
  {\attribute{id}{*}} attribute.}

\atabelt{format}{use}{cmml, default, html, mathematica, pmml, TeX,}
 {specifies the output format for which the notation is specified}

\atabelt{globals}{path-just}{}
 {points to the {\tt{theory-inclusion}s} that is the rest of the inclusion path.}

\atabelt{height}{omlet}{}
 {specifies the height of the rectangle on the screen taken up by the results of an
  {\tt{omlet}}}

\atabelt{hiding}{imports}{}
 {specifies the names of symbols that are not imported from the source theory}

\atabelt{href}{data}{}
 {a URI to an external file containig the data.}

\atabelt{id}{}{}
 {associates a unique identifier to an element, which can thus be referenced 
  by an {\attribute{for}{*}} attribute.}

\atabelt{induced-by}{obligation}{}
 {points to the statement in the source theory that induces this proof obligation}

\atabelt{just-by}{definition, alternative}{}
 {points to an assertion that states the well-definedness or termination condition
  of a definition or the equivalence condition of an alternative definition.}

\atabelt{kind}{symbol}{object, sort, type}
 {specifies the kind of object defined in this declaration.}

\atabelt{links}{decomposition}{}
 {specifies a list of theory-  or axiom-inclusions that justify (by decomposition)
 the {\tt{theory-inclusion}} specified  in the {\tt{for}} attribute.}

\atabelt{local}{path-just}{}
 {points to the {\tt{axiom-inclusion}} that is the first element in the path.}

\atabelt{logic}{FMP}{{\rm token}}
 {specifies the logical system used to encode the property.} 

\atabelt{name}{OMS, OMV}{}
 {the name of a symbol or variable.}

\atabelt{omdoc}{loc}{}
 {specifies the location of the {\omdoc} document containing the theory.}

\atabelt{omdoc-element}{presentation}{}
 {specifies the {\omdoc} element the presentation information applies to.}

\atabelt{parent}{presentation}{OMA, OMATTR,OMBIND}
 {specifies the parent element of the symbol for which notation information is 
  specified}

\atabelt{precedence}{presentation}{}
 {the precedence of a function symbol (for elision of brackets)}

\atabelt{proofs}{assertion}{}
 {specifies a list of URIs to proofs of this assertion.}

\atabelt{pto, pto-version}{private, code}{}
 {specifies the system and its version this data or code is private to}

\atabelt{replaces}{private}{}
 {points to a set of  elements whose content is replaced by the content 
  of the {\tt{private}} element for the system.}

\atabelt{requires}{private, code, use}{}
 {points to a {\tt{code}} element that is needed for the execution of this data by
  the system.}

\atabelt{rbrack}{presentation, use}{}
 {the right bracket  to use in the notation of a function symbol}

\atabelt{role}{Creator, Collaborator}
 {aft, ant, aqt, aui, aut, clb, edt, ths, trc, trl}
 {the MARC relator code for the contribution of the individual.}

\atabelt{size}{data}{}
 {specifies the size the data specified by a {\tt{data}} element. The value should
  be number of kilobytes}

\atabelt{scope}{symbol}{global, local}
 {specifies the visibility of the symbol declared. This is a very crude
  specification, it is better to use theories  and importing to specify symbol 
  accessibility.}

\atabelt{separator}{presentation, use}{}
 {the separator for the arguments to use in the notation of a function symbol}

\atabelt{sort}{argument}{}
 {specifies the argument sort of the constructor}

\atabelt{style}{*}{}
 {specifies a token for a presentation style to be picked up in a
 {\tt{presentation}} element.}

\atabelt{system}{use}{pres, xsl}
 {The transformation system to be used for specification. Use {\tt{'pres'}} for the
 {\omdoc} metalanguage, and {\tt{'xsl'}} for straight {\xslt}.}

\atabelt{system}{type}{}
 {A token that specifies the logical type system that governs the type specified
 in the type element.}

\atabelt{theory}{*}{}
 {specifies the home theory of an {\omdoc} statement.}

\atabelt{theory}{loc}{}
 {specifies the theory the {\element{loc}} element locates}

\atabelt{to}{theory-inclusion, axiom-inclusion}{}
 {specifies the target theory}

\atabelt{total}{selector}{no, yes}
 {specifies whether the symbol declared here is a total or partial function.}

\atabelt{type}{adt}{free, generated, loose}
 {defines the semantics of an abstract data type {\tt{free}} = no junk, no confusion,
 {\tt{generated}} = no junk, {\tt{loose}} is the general case.}

\atabelt{type}{asssertion} 
 {theorem, lemma, corollary, conjecture, false-conjecture,
  obligation, postulate, formula, assumption, proposition }
 {tells you more about the intention of the assertion}

\atabelt{type}{definition}{implicit, inductive, obj, recursive, simple}
 {specifies the definition principle}

\atabelt{type}{example}{against, for}
 {specifies whether the objects in this example support or falsify some conjecture}

\atabelt{type}{imports}{global, local}
 {{\tt{local}} imports only concern the assumptions directly stated in the
   theory. {\tt{global}} imports also concern the ones the source theory inherits.}

\atabelt{type}{omgroup, omdoc}
 {alternative, contrast, narrative, dataset, datalabels, datadata, theory-collection}
 {the first three give the text category, the second three are used for
                        generalized tables, and the last one for collections of theory.}

\atabelt{type}{omlet}{js, image}
 {the type of an omlet, e.g. {\tt{'image'}}}

\atabelt{type}{omtext}
 {abstract, antithesis, comment, conclusion, elaboration, evidence, 
  introduction,  motivation, thesis}
 {a specification of the intention of the text fragment, in reference to context.}

\atabelt{verdict}{answer}{}
 {specifies the truth or falsity of the answer. This can be used e.g. 
  by a grading application.} 

\atabelt{version}{omdoc}{1.1}
 {specifies the version of the document, so that the right DTD is used}

\atabelt{via}{inclusion}{}
 {points to a theory-inclusion that is required for an actualization}

\atabelt{width}{omlet}{}
 {specifies the width of the rectangle on the screen taken up by the results of an
  {\tt{omlet}}}

\atabelt{xml:lang}{*}{}
 {the language the text in the element is expressed in. 
  This must be a RFC-639 compliant specification of the primary language of the content.} 

\atabelt{xmlns}{omdoc}{{\url{http://www.mathweb.org/omdoc}}}
 {fixes the {\omdoc} namespace}

\atabelt{xref}{*}{ref, method, premise, presentation, {\rm some {\openmath} elements}}
 {a uniform resource identifier ({\ttin{URI}}) used for cross-referencing. The
  element, this URI points to should be in the place of the object containing this
  attribute.} 
\end{longtable}}

\chapter{The {\ifpdf{OMDoc}\else{\omdoc}\fi} Document Type Definition}\label{sec:dtd}
We reprint the current version of the {\omdoc} {\indextoo{document type
    definition}}. The original can be found at
{\url{http://www.mathweb.org/omdoc/dtd/omdoc.dtd}}. 

The DTD can be referenced by the {\indextoo{public
    identifer}}\index{identifier!public} {\tt{-//OMDoc//DTD OMDoc V1.1//EN}}. Thus
documents that use it have the {\indextoo{document type
    declaration}}\index{declaration!document type}

{\scriptsize\begin{boxedverbatim}
<!DOCTYPE omdoc Public "-//OMDoc//DTD OMDoc V1.1//EN" 
                       "http://www.mathweb.org/omdoc/omdoc.dtd"> 
\end{boxedverbatim}
}
in the preamble of the document. 

The {\indextoo{document type definition}}\index{definition!document type} includes
a variant document type definition for {\openmath} objects that differs from the
original (see~\url{http://www.openmath.org}) in that it allows to represent {\openmath} objects as
directed acyclic graphs. This extension is licensed by the OpenMath Standard that
says that any extension, from which valid OpenMath can be directly be generated,
is allowed.

{\scriptsize\listinginput[5]{1}{../../dtd/omdoc.dtd}}
\end{appendix}

%%% Local Variables: 
%%% mode: latex
%%% TeX-master: "omdoc"
%%% End: 
