\chapter{Mathematical Markup Schemes}\label{chap:markup}

Mathematical texts are usually very carefully designed to give them a structure
that supports understanding of the complex nature of the objects discussed and the
argumentations about them.  Of course this holds not only for texts in pure
mathematics, but for any argumentative text that contains mathematical notation,
in particular for texts from the sciences and engineering disciplines. In such
texts the document is often structured according to the argument made and
specialized notation (mathematical formulae) is used for the particular objects
discussed.  In contrast to this, the structure of texts like novels or poems
normally obey different (often esthetic) constraints. Therefore, we will use the
adjective ``mathematical'' in an inclusive way to make this distinction on text
form, not strictly on the scientific labeling.

The observation, that the task of recovering the semantic structure from the given
representation as a written text or a recording is central to understanding, holds
for any discourse. For mathematical discourses the structure is so essential that
the field has developed a lot of conventions about document form, numbering,
typography, formula structure, choice of glyphs for concepts, etc.  These
conventions have evolved over a long scientific history and carry a lot of the
information needed to understand a particular text. However, these conventions
were developed for the consumption by humans (mathematicians) and mainly with
``{\indextoo{ink-on-paper}}'' representations (books, journals, letters) in mind.

In the age of Internet publication and mathematical software systems the
``ink-on-paper'' target turns out to be too limited in many forms. The universal
accessibility of the documents on the Internet breaks the assumption implicit in
the design of traditional mathematical documents, that the reader will come from
the same (scientific) background as the author and will directly understand the
notations and structural conventions used by the author.  We can also rely less
and less on the assumption that mathematical documents are primarily for human
consumption as mathematical software systems are more and more embedded into the
process of doing mathematics. This, together with the fact that mathematical
documents are primarily produced and stored on computers, has led to the
development of specialized {\indextoo{markup}} schemes for mathematics.

In the next sections we will discuss some of the paradigmatic markup schemes
setting the stage with general document markup schemes for web-deployed documents.
In section~\ref{sec:math-objects} we will discuss representation formalisms for
mathematical objects. We will use section~\ref{sec:meta-math} to show that
extending general document markup approaches with mathematical formulae is not
sufficient for a content-based markup of mathematical documents, as it leaves many
central aspects of mathematical knowledge and structure implicit.

\section{Document Markup for the Web}\label{sec:markup-web}

In this section we will discuss some of the paradigmatic markup schemes to get a
feeling for the issues involved. Of course, we will over-stress the issues for
didactic reasons; due to economic pressures, none of the markup schemes survives
in a pure form anymore.

Text processors and desktop publishing systems (think for example of
{\indextoo{Microsoft}} Word\index{Word!Microsoft}) are software systems aiming to
produce ``{\emin{ink-on-paper}}'' or ``{\emin{pixel-on-screen}}'' representations
of documents.  They are very well-suited to execute the typographic conventions
mentioned above. Their internal markup scheme mainly defines presentation traits
like character position, font choice, and characteristics, or page breaks. This is
perfectly sufficient for producing high-quality presentations of the documents on
paper or on screen, but does not support for instance document reuse (in other
contexts or across the development cycle of a text). The problem is that these
approaches concentrate on the {\emin{form}} and not the {\emin{function}} of text
elements.  Think e.g. of the notorious section renumbering problems in early
({\indextoo{WYSIWYG}}) text processors. Here, the text form of a numbered section
heading was used to express the function of identifying the position of the
respective section in a sequence of sections (and maybe in a larger structure like
a chapter).

This perceived weakness has lead to markup schemes that concentrate more on
function than on form. We will take the
{\index{latex@\protect{\LaTeX}}\index{tex@\protect{\TeX}}}
{\TeX}/{\LaTeX}~\cite{Knuth:ttb84,Lamport:ladps94} approach as a paradigmatic
example here. A typical section heading would be specified by something like this:
\begin{verbatim}
\section[{\TeX}]{The Joy of {\indextoo{\TeX}}}\label{sec:TeX}
\end{verbatim}
This specifies the function of the text element: The title of the section should
be ``The Joy of {\TeX}'', which (if needed e.g. in the table of contents) can be
abbreviated as ``{\TeX}'', the word ``{\TeX}'' is put into the index, and the
section number can be referred to using the label {\tt{sec:TeX}}. To determine
from this functional specification the actual form (e.g. the section number, the
character placement and font information), we need a document formatting engine,
such as Donald {\indextoo{Knuth}}'s {\TeX} program~\cite{Knuth:ttb84}, and various
{\defin{style}} declarations, e.g. in the form of {\LaTeX} style
files~\cite{Lamport:ladps94}\index{style!file}\index{file!style}. This program
will transform the functional specification using the style information into a
markup scheme that specifies the form, like
{DVI\index{dvi@DVI}}~\cite{Knuth:ttb84}, or
{\postscript\index{postscript@\postscript}}~\cite{Reid:plpd87} that can directly
be presented on paper or on screen. Note that e.g.  {\indextoo{renumbering}} is not
a problem in this approach, since the actual numbers are only inferred by the
formatter at runtime. This, together with the ability to simply change style file
for a different context, yields much more manageable and reusable documents, and
has led to a wide adoption of the function-based approach. So that even
word-processors like MS Word now include functional elements. Purely form-oriented
approaches like {DVI\index{dvi@DVI}} or
{\postscript\index{postscript@\postscript}} are normally only used for document
delivery.

To contrast the two markup approaches we will speak of {\defin{presentation
    markup}}\index{markup!presentation} for markup schemes that concentrate on
form and of {\defin{content markup}}\index{markup!content} for those that specify
the function and infer the form from that. As we have emphasized before, few
markup schemes are pure in the sense of this distinction, for instance {\LaTeX}
allows to specify traits such as font size information, or using
\begin{center}
  {\verb+{\bf proof}:+}\ldots{\verb+\hfill\Box+}
\end{center}
to indicate the extent of a proof (the formatter only needs to ``copy'' them to
the target format). The general experience in such mixed markup schemes is that
presentation markup is more easily specified, but that content markup will enhance
maintainability, and reusability. This has led to a culture of style file
development (specifying typographical and structural conventions), which now gives
us a wealth of style options to choose from in {\LaTeX}.

Another member of the content markup family that additionally takes the problem of
document {\defin{metadata}} into account, i.e.  the description of the document
itself and the relations to other documents (cf. section~\ref{sec:metadata}), is
the ``{\indextoo{Simple Generalized Markup Language}}''
{\sgml}~\cite{Goldfarb:sgml90}.  It tries to give the markup scheme a more
declarative semantics (as opposed to the purely procedural -- and rather baroque
-- semantics of {\TeX}), to make it simpler to reason about (and thus reuse)
documents. It comes with its own style sheet language
{\dsssl\index{dsssl@\dsssl}}~\cite{DuCharme:fddsj97} and formatter Jade.

The Internet, where screen presentation, hyperlinking, computational limitations,
and bandwidth considerations are much more important than in the ``ink-on-paper''
world of publishing, has brought about a whole new set of markup schemes. The
problems that need to be addressed are that 
\begin{enumerate}
\item[$i)$] the size, resolution, and color depth of a given screen are not known
  at the time the document is marked up,
\item[$ii)$] the structure of a text is no longer limited to a linear text with
  (e.g. numbered) cross-references as in a book or article as Internet documents
  are in general hypertexts,
\item [$iii)$] the computational resources of the computer driving the screen are
  not known beforehand. Therefore the distribution of work (e.g. formatting steps)
  between the client and the server has to be determined at runtime. Finally, the
  related problem that
\item[$iv)$] the bandwidth of the Internet is ever-growing but limited.
\end{enumerate}

The ``{\indextoo{Hypertext Markup Language}}''
({\html}\index{html@\html}~\cite{RagHor:html98}) is a presentation markup scheme
that shares the basic syntax with {\sgml} and addresses the problem of variable
screen size and hyperlinking by exporting the decision of character placement and
page order to a {\indextoo{browser}} running on the client.  This ensures the high
degree of reusability of documents on the Internet, while conserving bandwidth, so
that {\html} carries most of the markup on the Internet today. Of course {\html}
has been augmented with its own (limited) style sheet language
{CSS\index{css@CSS}}~\cite{BosHak:css98} that is executed by the browser. The need
for content markup schemes for maintaining documents on the server, as well as for
specialized presentation of certain text parts (e.g. for mathematical or chemical
formulae), has led to a profusion of markup schemes for the Internet, most of
which share the basic {\sgml} syntax with {\html}. However, due to its origin in
the publishing world, full {\sgml} is much too complex for the Internet, and in
particular the {\dsssl\index{dsssl@\dsssl}} formatter is too unwieldy and
resource-hungry for integration into web browsers.

\section{{\ifpdf{XML}\else{\xml}\fi}, the eXtensible Markup Language}\label{sec:xml}

This diversity problem has led to the development of the unifying
{\xml}{\index{xml@{\xml}}} (eXtensible Markup Language)
framework~\cite{bray:XML97} for Internet markup languages, which we will introduce
in more detail in this section. As {\omdoc} and all mathematical markup schemes
discussed here are {\xml} applications (instances of the {\xml}
framework)\index{xml@{\xml} application}\index{application!{\xml}}, we will go
more into the technical details to supply the technical prerequisites for
understanding the specification. We will briefly mention {\xml} validation and
transformation tools. Readers with prior knowledge of {\xml} can safely skip this
section, if the material reviewed in this section is not enough, we refer the
reader to~\cite{Harold:xb01}.

Conceptually speaking, {\xml} views a document as a tree\index{document
  tree}\index{tree!document} of so-called {\defin{elements}}.  For communication
this tree is represented as a well-formed bracketing structure (see
{\myfigref{om-commutativity}} for an example), where the brackets of an element
{\tt{el}} are represented as {\tt{<el>}} (opening) and {\tt{</el>}} (closing); the
leaves of this tree are represented as {\defins{empty
    element}}\index{element!empty} {\tt<el></el>}, which can be abbreviated as
{\tt<el/>}\index{el@{\tt<el>},{\tt</el>},{\tt<el/>}}. The element nodes of this
tree can be annotated by further information in so-called {\defins{attribute}} in
opening brackets: {\tt<el visible="no">} might add the information for a
formatting engine to hide this element. As a document is a tree, the {\xml}
specification mandates that there must be a unique {\defin{document root}}, which
in {\omdoc} is the {\element{omdoc}} element.  Note that all {\xml} parsers will
reject a document that is not well-formed {\xml}, e.g. if it contains non-matching
element brackets (e.g. a single {\tt{<br>}}) or multiple document roots.

{\xml} offers two main mechanisms for specifying a subset of trees (or
well-bracketed {\xml} documents) as admissible.  A {\defin{document type
    definition}} {\defin{DTD}} is a {\indextoo{context-free
    grammar}}\index{grammar!context-free} for trees\footnote{Actually, a recent
  extension of the {\xml} standard ({\xlink}) also allows to express graph
  structures, but the admissibility of graphs is not covered by the DTD. See also
  section~\ref{sec:catalogue} on cross-referencing in {\omdoc}.}, that can be used
by {\defins{validating {\xml} parser}} to reject {\xml} documents that do not
conform to the {\omdoc} DTD (cf.  appendix~\ref{sec:dtd}).  Note that DTDs cannot
enforce all constraints that a particular {\xml} application may want to impose on
documents. Therefore DTD validation is only a necessary condition for
{\defin{validity}} with respect to that application.  Recently {\xml} has added
another grammar formalism: the {\xml} schema\index{xml@{\xml}
  schema}\index{schema!{\xml}} language, which can express a slightly stronger set
of constraints. Since an {\xml} schema allows stronger document validation, it
usually takes {\indextoo{normative precedence}}\index{precedence!normative} over the
DTD in specifications.

\begin{myfig}{xml-dtd}{The Structure of an {\xml} document with DTD.}\footnotesize
\begin{boxedverbatim}
<?xml version="1.0"?> 
<!DOCTYPE omdoc PUBLIC "-//OMDoc//DTD OMDoc V1.1//EN" 
                       "http://www.mathweb.org/omdoc/omdoc.dtd" []> 
<omdoc>
 ...  
</omdoc>
\end{boxedverbatim}
\end{myfig}
Concretely, an {\omdoc} document has the general form shown in
{\myfigref{xml-dtd}}. The first line identifies the document as an {\xml} document
(version 1.0 of the {\xml} specification). The second line specifies the DTD and
the document root {\omdoc} this it is intended for. In this case the
{\element{omdoc}} element starting in line three is the root element and will be
validated against the DTD found at the URL specified in line two. The last line
contains the end tag of the {\element{omdoc}} element and ends the file. Every
{\xml} element following this line would be considered as another document root.

\begin{myfig}{internal}{A Document Type Declaration with Internal Subset}
\small
\begin{boxedverbatim}
<!DOCTYPE omdoc PUBLIC "-//OMDoc//DTD OMDoc V1.1//EN" 
                       "http://www.mathweb.org/omdoc/omdoc.dtd" 
  [<!ENTITY % mathmldtd SYSTEM 
              "http://www.w3.org/Math/DTD/mathml1/mathml.dtd">
   %mathmldtd;
   <!ELEMENT el (math)><!ATTLIST el att CDATA #REQUIRED>]> 
\end{boxedverbatim}
\end{myfig}
A DTD specified in the {\tt{<!DOCTYPE}} declaration can be enhanced or modified by
adding declarations in the {\indextoo{internal subset}}\index{DTD!internal subset}
of the {\ttin{DOCTYPE}} declaration (the empty {\tt{[]}} in
{\myfigref{xml-dtd}}).\label{page:internal-dtd} In {\myfigref{internal}}, we have
modified the DTD by declaring that the {\tt{el}} element has a required attribute
{\tt{att}} and must contain a single {\element{math}} child. The declarations for
that are contained in the {\mathml} DTD, which we have included by first declaring
the {\indextoo{parameter entity}} {\tt{\%mathmldtd;}} and then referencing it. The
internal subset allows to change the DTD grammar for selected elements and to
extend the admissible content of elements that were given the {\indextoo{content
    type {\tt{ANY}}}}\index{ANY!content type} in the original DTD.

The {\xml} schema applicable to an {\xml} document is given by a different
mechanism. {\xml} assumes that all elements in a document belong to a given
{\indextoo{namespace}}\index{xml@{\xml} namespace}. Technically, an {\xml} namespace
is simply a string that uniquely identifies the intended semantics of the
elements.  It is a {\indextoo{URI}} ({\indextoo{uniform resource identifier}}; a
special string that identifies resources on the Internet, see~\cite{Harold:xb01}).
Note that it need not be a valid {\indextoo{URL}} ({\indextoo{uniform resource
    locator}}; i.e.  a pointer to a document provided by a web server.).
Namespaces are used to differentiate {\xml} vocabularies or languages, so they can
be safely mixed in documents. In principle, every {\indextoo{element}} and
{\indextoo{attribute}} name is prefixed by a namespace, i.e. it is a pair
{\tt{ns:n}}, where {\tt{ns}} is a namespace and {\tt{n}} is a simple name (that
does not contain a colon). We call such a namespace/name pair a {\indextoo{qualified
    name}}\index{name!qualified}. In most cases, namespaces can be elided or
abbreviated when writing {\xml}.  Namespaces can be declared on any {\xml} element
via the {\attribute{xmlns}{*}} attribute: the element and all its descendents are
in this namespace, unless they have a namespace attribute of their own or there is
a namespace declaration in a closer ancestor that overwrites it.  Similarly, a
{\indextoo{namespace abbreviation}}\index{abbreviation!namespace} can be declared on
any element, it is declared by an attribute declaration of the form
{\tt{xmlns:nsa="nsURI"}}, where {\tt{nsa}} is a name space abbreviation, i.e. a
simple name and {\tt{nsURI}} is the URI of the namespace.  In the scope of this
declaration (in all descendants, where it is not overwritten) a qualified name
{\tt{nsa:n}} denotes the qualified name {\tt{nsURI:n}}.

\begin{myfig}{xml-schema}{An {\xml} document with {\xml} Schema.}\footnotesize
\begin{boxedverbatim}
<?xml version="1.0"?>
<omdoc xmlns="http://www.mathweb.org/omdoc" 
       xmlns:xsi="http://www.w3.org/2001/XMLSchema-instance" 
       xsi:schemaLocation="omdoc.xsd http://www.mathweb.org/omdoc">
 ...
</omdoc>
\end{boxedverbatim}
\end{myfig}
Let us now consider {\myfigref{xml-schema}}, which shows the attributes,
namespaces, and namespace abbreviations necessary to associate an {\xml} document
with an {\xml} schema. The {\attribute{xmlns}{omdoc}} attribute in the
{\element{omdoc}} element declares that the URI
{\url{http://www.mathweb.org/omdoc}} is the default namespace for the document,
i.e. all element and attribute names without a colon are in this namespace. The
attribute {\tt{xmlns:xsi}} declares the namespace abbreviation {\ttin{xsi}} for
the namespace of {\xml} {\indextoo{schema instance}s}\index{instance!{\xml} schema}.
Finally, the attribute {\attribute{xsi:schemaLocation}{omdoc}} identifies the {\xml}
schema that is relevant for this element (and thus for the document). Note that
with this mechanism schemata can be associated with elements (in contrast to DTDs
that can only be associated with whole documents), which makes mixing {\xml}
vocabularies much simpler.

Since {\xml} elements only encode trees, the distribution of {\indextoo{whitespace}}
(including {\index{line-feed}s}) in non-text elements has no meaning in {\xml}, and
can therefore be added and deleted without effecting the semantics.

{\xml} considers as comments\index{comment!{\xml}}\index{xml@{\xml}!comment}
anything between {\verb+<!--+} and {\verb+-->+} in a document. They should be used
with care, since they are not even read by the {\xml}
parser\index{xml@{\xml}!parser}\index{parser!{\xml}}, and therefore do not survive
processing by {\xml} applications. Material that is relevant to the document, but
not valid {\xml}, e.g. binary data or data that contains angle brackets or
elements that are unbalanced or not defined in the DTD can be embedded into
{\ttin{CDATA}} sections. A {\ttin{CDATA}} section begins with {\verb+<[CDATA[+}
and suspends the {\xml} parser until the string {\verb+]]>+} is found. Another way
to include such material is to escape\index{escaping@{\xml}
  escaping}\index{xml@{\xml}!escaping} the {\xml}-specific symbols ``{\verb+<+}'',
``{\verb+>+}'', and ``{\verb+&+}'' to {\ttin{\&lt;}}, {\ttin{\&gt;}} and
{\ttin{\&amp;}}. According to the {\xml} specification a {\ttin{CDATA}} section is
equivalent to directly including the {\xml}-escaped contents. For instance
{\verb+<[CDATA[a<b<sup>3</sup>]]>+} and {\verb+a&lt;b&lt;sup&gt;3&lt;/sup&gt;+}
are equivalent, as a consequence an {\xml} application is free to choose the form
of its output and the particular form should not relied upon.


{\xml} comes with the {\indextoo{{\xslt}}} {\indextoo{style}} language
transformations~\cite{Deach:exls99}, that is lightweight enough to allow
integration of {\xslt}-transformers into browsers (they are present in version 6
of {\indextoo{Microsoft}}'s {\indextoo{Internet Explorer}} and in version 6 of the
{\indextoo{Netscape}} {\indextoo{Navigator}}).  {\xslt} programs or
{\indextoo{style sheet}s} consist of a set of so-called {\indextoo{templates}}
(rules that match certain nodes in the {\xml} tree) that are recursively applied
to the input tree to produce the desired output. 

\section{Mathematical Objects and Formulae}\label{sec:math-objects}

The two best-known open markup formats for representing mathematics for the Web
are {\mathml} and {\openmath}. There are various other formats that are
proprietary or based on specific mathematical software packages like
{\indextoo{Wolfram Research}}'s {\mathematica\index{Mathematica@\mathematica}}. We
will not concern ourselves with them, since we are only interested in open
formats.

{\defemph{\mathml}}~\cite{CarIon:MathML01}\index{mathml@{\mathml}} is
an {\xml}-based markup scheme for mathematical formulae. It has
developed out of the effort to include presentation primitives for
mathematical notation (in {\TeX}{\index{tex@{\TeX}}} quality) into
{\html}, and was the first {\xml} application.  Since the aim is to do
most of the formatting inside the browser, where resource
considerations play a large role, it restricts itself to a fixed set
of mathematical concepts -- the so-called {\indextoo{K-12}} fragment
of mathematics (Kindergarten to $12^{th}$ grade). K-12 is a large set
of commonly used glyphs for mathematical symbols and very general and
powerful presentation primitives, as they make up the lower level of
{\TeX}.  However it does not offer the programming language features
of {\TeX}\footnote{{\TeX} contains a full,
{\indextoo{Turing}}-complete -- if somewhat awkward -- programming
language that is mainly used to write {\indextoo{style file}}s.  This
is separated out by {\mathml} to the {\xslt} language it inherits from
{\xml}.} for the obvious computing resource considerations.  {\mathml}
is supported by the current versions of the primary commercial
browsers {\msie}\index{internet@{\msie}} and
{\netscape}\index{netscape@{\netscape}} by special plug-ins, and
natively by {\mathml}-enabled versions of the open source browsers
{\mozilla}\index{mozilla@{\mozilla}} and {\sc
Amaya}\index{amaya@{\sc Amaya}}.

\begin{myfig}{pcmml}{Mixing Presentation and Content {\mathml}}\footnotesize
\begin{boxedverbatim}
<semantics>
 <mrow>
  <mrow><mo>(</mo><mi>a</mi> <mo>+</mo> <mi>b</mi><mo>)</mo></mrow>
  <mo>&InvisibleTimes;</mo>
  <mrow><mo>(</mo><mi>c</mi> <mo>+</mo> <mi>d</mi><mo>)</mo></mrow>
 </mrow>
 <annotation-xml encoding="MathML-Content">
  <apply><times/>
   <apply><plus/><ci>a</ci> <ci>b</ci></apply>
   <apply><plus/><ci>c</ci> <ci>d</ci></apply>
  </apply>
 </annotation-xml>
</semantics>
\end{boxedverbatim}
\end{myfig}

{\mathml} also offers content markup for mathematical formulae, a sub-language
called {\indextoo{content {\mathml}}} to contrast it from the
{\indextoo{presentation {\mathml}}} described above. Furthermore, it offers a
specialized {\element{semantics}} element that allows to annotate {\mathml}
formulae with content markup, e.g.  so that they can be passed on to other
mathematical software systems like computer algebra systems. {\Myfigref{pcmml}}
shows an example of this for the arithmetical expression $(a+b)(c+d)$. The
outermost {\element{semantics}} element is a {\mathml} primitive for annotating
{\mathml} elements with other representations. Here it is used for mixing
presentation and content markup. The first child of the {\element{semantics}}
element is the presentation (this is used by the {\mathml}-aware browser) which is
annotated by {\element{annotation-xml}} element, which contains the content
markup. Let us first look at the presentation markup. The {\element{mrow}}
elements are a general grouping device the layout engine uses for purposes of
alignment and line-breaking. The {\element{mo}} elements marks its content as a
{\indextoo{mathematical operator}}\index{operator!mathematical} and the
{\element{mi}} element marks its content as a {\indextoo{mathematical
    identifier}}\index{identifier!mathematical}. The {\indextoo{entity
    reference}}\index{reference!entity} {\tt{{\&}InvisibleTimes;}} is a
character that is not displayed, but stands for the multiplication operator.


For content markup, the logical structure of the formula is in the center.
{\mathml} uses the {\element{apply}} element for function application. In this case the
multiplication function {\element{times}}, which is applied to the
results of the addition
function {\element{plus}}, applied to some identifiers. Both the
elements {\element{times}} and {\element{plus}} are modeled as empty elements. Note that brackets
are not explicitly represented, since they are purely presentational devices and
the information is implicit in the structure of the formula and can be deduced
from notational conventions.  The {\element{mi}} element has content counterpart
{\element{ci}} for {\indextoo{content identifier}}\index{identifier!content},
which conceptually corresponds to a {\indextoo{logical
    variable}}\index{variable!logical}. The concept of a {\indextoo{domain
    constant}s}\index{constant!domain} is either modeled by a special element (if
it is in the K-12 range as {\element{plus}} and {\element{times}}, there are about
80 others) or by the {\element{csymbol}} element.

In contrast to this very rich language that defines the meaning of extended
presentation primitives, the {\openmath} standard~\cite{CapCoh:doms98} builds on
an extremely simple kernel (mathematical objects represented by content formulae),
and adds an extension mechanism, the so-called {\defemph{content
    dictionaries}}\index{content!dictionary}\index{dictionary!content}. These are
{\indextoo{machine-readable}} specifications of the meaning of the mathematical
concepts expressed by the {\openmath} symbols.  Just like the library mechanism of
the {\ttin{C}} programming language, they allow to externalize the definition of
extended language concepts. As a consequence, K-12 need not be part of the
{\openmath} language, but can be defined in a set of content dictionaries (see
{\url{http://www.openmath.org/cdfiles/html/core}}). Moreover, {\openmath} is
purely based on content markup.

The central construct of {\openmath} is that of an {\indextoo{{\openmath}
    object}}\index{object!{\openmath}} ({\eldef{OMOBJ}}), which has a tree-like
representation made up of {\indextoo{application}s} ({\eldef{OMA}}),
{\indextoo{binding structure}s} ({\eldef{OMBIND}} using {\eldef{OMBVAR}} to tag the
{\indextoo{bound variable}s}\index{variable!bound}), {\indextoo{variable}s}
({\eldef{OMV}}) and {\indextoo{symbol}s} ({\eldef{OMS}}). The {\element{OMS}}
element carries attributes {\attribute{cd}{OMS}} and {\attribute{name}{OMS}}
attributes.  The {\attribute{name}{OMS}} attribute gives the name of the symbol.
The {\attribute{cd}{OMS}} attribute specifies {\indextoo{content dictionary}}, a
document that defines the meaning of a collection of symbols including the one
referenced by the {\element{OMS}} itself. As variables do not carry a meaning
independent of their local content, {\element{OMV}} only carries a
{\attribute{name}{OMV}} attribute. See {\myfigref{om-commutativity}} for an
example that uses most of the elements.

For convenience, {\openmath} also provides other basic data types useful in
mathematics: {\eldef{OMI}} for {\indextoo{integer}s}, {\eldef{OMB}} for
{\indextoo{byte array}s}, {\eldef{OMSTR}} for {\indextoo{string}s}, and {\eldef{OMF}}
for floating point numbers, and finally {\eldef{OME}} for {\indextoo{error}s}.  Just
like {\mathml}, {\openmath} offers an element for annotating (parts of) formulae
with external information (e.g. {\mathml} or {\LaTeX} presentation): the
{\eldef{OMATTR}}\footnote{Note that the meaning of this element is somewhat
  underdefined, it is stated in the standard, that any {\openmath} compliant
  application is free to disregard attribuitions (so they do not have a meaning),
  but in practice, they are often used for specifying e.g.  type information.}
element, which pairs an {\openmath} object with an attribute-value list. To
attribute an {\openmath} object, it is embedded as the second child in an
{\element{OMATTR}} element. The attribute-value list is specified by children of
the {\element{OMATP}} element, which is the first child, and has an even number of
children: children at even position must be {\element{OMS}} (specifying the
attribute), and children at odd positions are the values of the attributes given
by their immediately preceding siblings.

The content dictionaries that make up the extension mechanism provided in
{\openmath} are tied into the object representation by the {\attribute{cd}{OMS}}
attribute of the {\element{OMS}} element that specifies the defining content
dictionary.

{\openmath} and {\mathml} are well-integrated:
\begin{itemize}
\item the core content dictionaries of {\openmath} mirror the {\mathml} constructs
  (see {\url{http://www.openmath.org/cdfiles/html/core}}); there are converters
  between the two formats.
\item {\mathml} supports the {\element{semantics}} element, that can be used to
  annotate {\mathml} presentations of mathematical objects with their {\openmath}
  encoding. Analogously, {\openmath} supports the {\ttin{presentation}} symbol in
  the {\element{OMATTR}} element, that can be used for annotating with {\mathml}
  presentation.
\item {\openmath} is the designated extension mechanism for {\mathml} beyond K-12
  mathematics: content {\mathml} supports the {\element{csymbol}} element, which
  has an attribute {\attribute{definitionURL}{csymbol}} that points to a document (an
  {\openmath} CD) that defines the meaning of the symbol. The content of the
  {\element{csymbol}} element is {\mathml} presentation markup for the symbol.
\end{itemize}

\setbox0=\hbox{\scriptsize\begin{boxedverbatim}
<OMOBJ>                            
 <OMBIND>                          
  <OMS cd="quant1" name="forall"/> 
  <OMBVAR>                         
   <OMATTR>                        
    <OMATP>                        
     <OMS cd="sts" name="type"/>   
     <OMS cd="setname1" name="R"/>  
    </OMATP>                       
    <OMV name="a"/>                
   </OMATTR>                        
   <OMATTR>                        
    <OMATP>                        
     <OMS cd="sts" name="type"/>   
     <OMS cd="setname1" name="R"/>  
    </OMATP>                       
    <OMV name="b"/>                
   </OMATTR>                        
  </OMBVAR>                        
   <OMA>                           
    <OMS cd="relation" name="eq"/> 
    <OMA>                          
     <OMS cd="arith1" name="plus"/>
     <OMV name="a"/>               
     <OMV name="b"/>               
    </OMA>                         
    <OMA>                          
     <OMS cd="arith1" name="plus"/>
     <OMV name="b"/>               
     <OMV name="a"/>               
    </OMA>                         
  </OMA>                           
 </OMBIND>                         
</OMOBJ>                           
\end{boxedverbatim}
}
\setbox1=\hbox{\scriptsize\begin{boxedverbatim}
<math>
 <apply>  
  <forall/>
  <bvar> 





   <ci type="real">a</ci>






   <ci type="real">b</ci>

  </bvar>
  <apply>
   <eq/>
   <apply>
    <plus/>
    <ci type="real">a</ci>
    <ci type="real">b</ci>
   </apply>
   <apply>
    <plus/>
    <ci type="real">b</ci>
    <ci type="real">a</ci>
   </apply>
  </apply>
 </apply>
</math>
\end{boxedverbatim}
}

\begin{myfig}{om-commutativity}{$\forall a,b:{\bf R}.a+b=b+a$. in {\openmath}
    and {\mathml} format}
\begin{tabular}{cc}
 {\large{\openmath}}  &  {\large\mathml}\\
 \box0 & \box1
\end{tabular}
\end{myfig}
{\Myfigref{om-commutativity}} shows {\openmath} and content {\mathml}
representations of the law of commutativity for addition on the reals (the logical
formula $\forall a,b:R.a+b=b+a$). The mathematical meaning of symbols (that of
applications and bindings is known from the folklore) is specified in a set of
content dictionaries\index{content dictionary}, which contain formal
({\element{FMP}} ``{\indextoo{formal mathematical property}}'') or informal
({\element{CMP}} ``{\indextoo{commented mathematical property}}'') specifications
of the mathematical properties of the symbols. For instance, the specification in
{\myfigref{arith1}} is part of the standard {\openmath} content dictionary
{\ttin{arith1.ocd}} for the elementary arithmetic operations. The content of the
{\element{FMP}} element is actually the {\openmath} object in the representation
on the left of {\myfigref{om-commutativity}}, we have abbreviated it here in the
usual mathematical notation, and we will keep doing this in the remaining
document: wherever an {\xml} element in a figure contains mathematical notation,
it stands for the corresponding {\openmath} element.
\setbox0=\hbox{\begin{minipage}{10.2cm}\footnotesize
\begin{alltt}
<CDDefinition>
 <Name>plus</Name> 
 <Description> 
  The symbol representing an n-ary commutative function plus.
 </Description> 
 <CMP> for all a,b | a + b = b + a </CMP>
 <FMP>\(\forall a,b.a+b=b+a\)</FMP> 
</CDDefinition>
\end{alltt}
\end{minipage}}
\begin{myfig}{arith1}{Part of the {\openmath} CD {\ttin{arith1}}.}
\fbox{\box0}
\end{myfig}

\section{Meta-Mathematical Objects}\label{sec:meta-math}

The mathematical markup languages {\openmath} and {\mathml} we have discussed in
the last section have dealt with mathematical objects and formulae.  This level of
support is sufficient for representing very established areas of mathematics like
K-12 high school math, where the meaning of concepts and symbols is totally clear,
or for the communication needs of symbolic computation services like computer
algebra systems, which manipulate and compute objects like equations or groups.
The formats either specify the semantics of the mathematical object involved in
the standards document itself ({\mathml}) or in a fixed set of generally
agreed-upon documents ({\openmath} content dictionaries\index{content
  dictionary}). In both cases, the mathematical knowledge involved is relatively
fixed. Eeven in the case of {\openmath}, which has an extensible library
mechanism, it is not in itself an object of communication (content dictionaries
are mainly background reference for the implementation of {\openmath} interfaces).

There are many areas of mathematics, where this level of support is insufficient,
because the mathematical knowledge expressed in definitions, theorems (stating
properties of defined objects), their proofs, and even whole mathematical theories
becomes the primary ``object'' of mathematical communication. We will call these
``objects'' {\defins{meta-mathematical object}}\index{objects{meta-mathematical}},
since they contain knowledge {\em{about\/}} mathematical objects. As a consequence
it is not the structure of the mathematical objects themselves, but the structure
of elements of mathematical knowledge and their interdependencies that is
communicated, between mathematicians.

Traditional mathematics has developed a rich set of conventions to mark up the
structure of mathematical
knowledge\index{knowledge!mathematical}\index{mathematical!knowledge} in
documents\index{document!mathematical}\index{mathematical!document}. For instance,
mathematical
statements\index{statement!mathematical}\index{mathematical!statement} like
theorems, definitions, and proofs like the ones in {\myfigref{fragment}} are
delimited by keywords (e.g. {\bf Lemma} and {\boexchen}) or by changes in text
font (claims are traditionally written in italics). We will collectively refer to
meta-mathematical objects like {\indextoo{axiom}s}, {\indextoo{definition}s},
{\indextoo{theorem}s}, and {\indextoo{proof}s} as {\defins{mathematical
    statement}}\index{statement!mathematical}, since they state properties of
mathematical objects.

\begin{myfig}{fragment}{A fragment of a traditional mathematical Document}
  \fbox{\begin{minipage}{10cm}
      {\bf Definition 3.2.5} (Monoid)\\
    A monoid is a semigroup $S=(G,\circ)$ with an element $e\in G$, 
    such that $e\circ x=x$ for all $x\in G$. $e$ is called a left 
    unit of a $S$.\\[1ex]
    
    {\bf Lemma 3.2.5}\\
    {\em A monoid has at most one left unit.}\\
    {\bf Proof}: We assume that there is another left unit $f$\ldots This
    contradicts our assumption, so we have proven the claim.\hfill\kasten
  \end{minipage}}
\end{myfig}

The large-scale structure of mathematical knowledge is mapped to informal groups
of mathematical statements called theories\index{theory}, and often mapped into
monographies\index{monography} (titled e.g. ``Introduction to Group Theory'') or
{\indextoo{chapter}s} and {\indextoo{section}s} in {\indextoo{textbook}s}. The
rich set of relations among such theories is described in the text, sometimes
supported by mathematical statements called {\indextoo{representation
    theorem}s}\index{theorem!representattion}. In fact, we can observe that
mathematical texts can only be understood with respect to a particular
mathematical context given by a theory which the reader can usually infer from the
document. The context can be given explicitly, e.g. by the title of a book such as
``Introduction to the Theory of Finite Groups'' or implicitly (e.g. by the
fact that the e-mail comes from a person that we know works on finite groups, and
we can see that she is talking about math).
  
Mathematical theories {\index{mathematical theory}}\index{theory!mathematical}
have been studied by meta-mathe\-ma\-ti\-cians and logicians in the search of a rigorous
foundation of mathematical practice. They have been formalized as collections of
symbol declarations\index{symbol!declaration} giving names to the mathematical
objects that are particular to the theory and logical formulae, which state the
laws governing the properties of the theory. A key research question was to
determine conditions for the consistency of mathematical theories. In inconsistent
theories (such that do not have models) all statements are vacuously
valid\footnote{A statement is valid in a theory, iff it is true for all models of
  the theory. If there are none, it is vacuously valid.}, and therefore,
only consistent theories make interesting statements about mathematical objects.
It is one of the key observations of meta-mathematics that more formulae can be
added without endangering consistency, if they can be proven from the formulae
already in the theory. As a consequence, consistency of a theory can be determined
by classifying the formulae into {\indextoo{theorem}s}, i.e. those that have a
proof, and {\indextoo{axiom}s} -- those that do not -- and examining consistency
of the axioms only.  Thus the role of proofs is twofold, they allow to push back
the assumptions about the world to simpler and simpler assumptions, and they allow
to test the model by deriving consequences of these basic assumptions that can be
tested against the data.
  
A second important observation is that new symbols together with axioms defining
their properties can be added to a theory without endangering consistency, if they
are of a certain restricted syntactical form. These so-called
{\indextoo{definitional form}s} mirror the various types of mathematical
{\indextoo{definition}s} (e.g. equational, recursive, implicit definitions).  This
leads to the so-called {\indextoo{principle of conservative
    extension}}\index{conservative extension}, which states that conservative
extensions to theories (by {\indextoo{theorem}s} and {\indextoo{definition}s}) are
safe for mathematical theories, and that possible sources for
inconsistencies\index{inconsistency} can be narrowed down to small sets of axioms.

Even though all of this has theoretically been known to (meta)-mathema\-ticians for
almost a century, it has only been an explicit object of formal study and
exploited by mathematical software systems in the last decades. Much of the
meta-mathematics has been formally studied in the context of proof development
systems like \index{{\sc Automath}} {\sc Automath}~\cite{Bruijn80}
Nuprl~\cite{Constable86}, HOL~\cite{GoMe93}, {\sc Mizar}~\cite{Rudnicki:aomp92}
and {\OMEGA}~\cite{BenzmuellerEtAl:otama97} which utilize strong logical systems
that allow to express both mathematical statements and proofs as mathematical
objects.  Some systems like Isabelle~\cite{Paulson90} and Elf~\cite{Pfenning91}
even allow the specification of the logic language itself, in which the reasoning
takes place. Such semi-automated theorem proving systems have been used to
formalize substantial parts of mathematics and mechanically verify many theorems
in the respective areas. These systems usually come with a library system that
manages and structures the body of mathematical knowledge formalized in the system
so far.

In software engineering, mathematical theories have been studied under the label
of (algebraic) {\indextoo{specification}}\index{algebraic specification}. Theories
are used to specify the behavior of programs and software components. Under the
pressure of industrial applications, the concept of a theory (specification) has
been elaborated from a practical point of view to support the structured
development of specifications, theory {\indextoo{reuse}}\index{theory!reuse}, and
modularization. Without this additional structure, real world specifications
become unwieldy and unmanageable in practice. Just as in the case of the theorem
proving systems, there is a whole zoo of specification languages, most of them
tied to particular software systems.  They differ in language primitives,
theoretical expressivity, and the level of tool support.

Even though there have been standardization efforts, the most recent one being the
{\casl} standard (Common Algebraic Specification Language; see~\cite{CoFI98})
there have been no efforts of developing this into a general markup language for
mathematics with attention to web communication and standards. The {\omdoc} format
attempts to provide a content-oriented markup scheme that supports all the aspects
and structure of mathematical knowledge we have discussed in this section. Before
we define the language in the next chapter, we will briefly go over the
consequences of adopting a markup language like {\omdoc} as a standard for
web-based mathematics.

\section{An Active Web of Mathematical Knowledge}\label{sec:mathweb}

It is a crucial -- if relatively obvious -- insight that true cooperation of
mathematical services is only feasible if they have access to a joint corpus of
mathematical knowledge. Moreover, having such a corpus would allow to develop
added-value services like
\begin{enumerate}
\item\label{service:cut-paste} cut and paste on the level of computation (take the
  output from a web search engine and paste it into a computer algebra system),
\item\label{service:proof-check} automatically proof checking published proofs,
\item\label{service:explain} math explanation (e.g. specializing a proof to an
  example that simplifies the proof in this special case),
\item\label{service:search} semantical search for mathematical concepts (rather
  than keywords),
\item\label{service:mine} data mining for representation theorems (are there
  unnoticed groups out there),
\item\label{service:classify} classification: given a concrete mathematical
  structure, is there a general theory for it?
\end{enumerate}
As the online mathematical knowledge is presently only machine-readable, but not
machine-understandable, all of these services can currently only be performed by
humans, limiting the accessibility and thus the potential value of the
information. Services like this will transform the now passive and human-centered
fragement of the Internet that deals with mathematical content, into an active (by
the services) web of mathematical knowledge (we will speak of {\indextoo{mathweb}}
for this vision).

Of course, this promise of activating a web of knowledge is in no way limited to
mathematics, and the task of transforming the current presentation-oriented
world-wide web into a ``semantic web''~\cite{BernersLee:tsw98} has been identified
as one of the main challenges by the world wide web consortium (W3C, the
fundamental standardizing body for the WWW, see {\url{http://www.w3c.org}}).

The direct applications of {\mathweb} (apart from the general effect towards a
semantic web) are by no means limited to mathematics proper. Until now, the {\mathml}
working group in the W3C has led the way in many web technologies (presenting
mathematics on the web taxes the current web technology to its limits); the
endorsement of the {\mathml} standard by the W3 Committee is an explicit testimony to
this. We expect that the effort of creating an infrastructure for digital
mathematical libraries will play a similar role, since mathematical knowledge is
the most rigorous and condensed form of knowledge, and will therefore pinpoint the
problems and possibilities  of the semantic web.

All modern sciences have a strongly mathematicised core, and will benefit. The
real market and application area for the techniques developed in this project lies
with high-tech and engineering corporations like Airbus Industries, Daimler
Chrysler, Phillips, ... that rely on huge formula databases.  Currently, both the
content markup as well as the added-value services alluded to above are very
underdeveloped, limiting the usefulness of the vital knowledge. The content-markup
aspect needed for mining this information treasure and obtaining a competitive
edge in development is exactly what we are attempting to develop in {\omdoc}.

%%% Local Variables: 
%%% mode: latex
%%% TeX-master: "omdoc"
%%% End: 
