\section*{1 Introduction}

No errata known.
\section*{2 Mathematical Markup Schemes}
No errata known.
\section*{3 OMDoc Elements}
No errata known.
\subsection*{3.1 Metadata for Mathematical Elements}
No errata known.

\subsection*{3.2 Mathematical Statements}

\subsubsection*{3.2.1 Specifying Mathematical Properties}
\begin{itemize}
\item The text fails to make clear for the
  {\attribute{id}{om:*}}/{\attribute{xref}{om:*}} to {\openmath} objects are only
  allowed, if the referencing element has the same name as the referenced one. In
  particular, there are no implicit conversions.
  
\item The text fails to mention the {\attribute{logic}{FMP}} attribute of the
  {\element{FMP}} element. We need to add ``{\element{FMP}}s always appear in
  groups, which can differ in the value of their {\attribute{logic}{FMP}}
  attribute, which specifies the logical formalism. The value of this attribute
  specifies the logical system used in formalizing the content. All members of the
  {\indextoo{multi-logic {\tt{FMP}} group}}\index{FMP group!multi-logic} have to
  formalize the same mathematical object or property, i.e. they have to be
  translations of each other.''
\end{itemize}

\subsubsection*{3.2.2 Symbols, Definitions, and Axioms}
\begin{enumerate}
\item The values {\attval{obj}{type}{definition}} and
  {\attval{simple}{type}{definition}} were overlapping, and the role of the
  {\element{FMP}} and {\element{OMOBJ}} children of the {\element{definition}} was
  unclear. The value {\attval{obj}{type}{definition}} has been dropped and we have
  clarified that in simple definitions, the {\element{OMOBJ}} is the substitution
  element whereas the {\element{FMP}} captures the meaning of the {\element{CMP}}
  group in Logic.
\item The attribute {\attribute{kind}{symbol}} of the {\element{symbol}} element
  can also have the value {\attval{sort}{kind}{symbol}} for sets that are
  inductively built up from constructor symbols
\end{enumerate}

\subsubsection*{3.2.3 Assertions and Alternatives}
No errata known.
\subsubsection*{3.2.4 Mathematical Examples in {\ifpdf{OMDoc}\else{\omdoc}\fi}}
No errata known.
\subsubsection*{3.2.5 Representing Proofs in {\ifpdf{OMDoc}\else{\omdoc}\fi}}
No errata known.
\subsubsection*{3.2.6 Abstract Data Types}
The optional {\element{recognizer}} element should be a child of
{\element{sortdef}}, and not of the {\element{constructor}} element. The
specification text and examples are correct, but the quick reference table is
incorrect.

\subsection*{3.3 Theories as Mathematical Contexts}
\subsubsection*{3.3.1 Simple Inheritance}
The content model for {\element{theory}} in Figure 3.22 is incorrect. It should
read {\tt{commonname*, CMP*, ({\rm statement} | inclusion, imports)*}}. Moreover,
the attribute model has a spurious comma. Furthermore, the text should make clear
that {\omdoc}1.1 does not allow theories to nest and that theories can include
{\element{imports}} statements.

\subsubsection*{3.3.2 Inheritance via Translations}
No errata known.

\subsubsection*{3.3.3 Statements about Theories}
The text does not make this clear, but the elements {\element{theory-inclusion}},
{\element{axiom-inclusion}} and {\element{decomposition}} may not occur in a
{\element{theory}} element. Worse, the document type definition allows this as
well. 

\subsubsection*{3.3.4 Parametric theories in {\ifpdf{OMDoc}\else{\omdoc}\fi}}
No errata known.

\subsection*{3.4 Auxiliary Elements}

\subsubsection*{3.4.1 Preservation of Text Structure}
\begin{description}
\item[Figure 3.29:] Specifying Tables with {\tt{<omgroup type="dataset">}} The first
label for the second axis is ``{\tt{b11}}''. Should be ``{\tt{b1}}''.
\item[omgroup] The DTD did not contain value {\attval{narrative}{type}{omgroup}} that was present in
  the specification.
\end{description}

\subsection*{3.4.4 Exercises}
In the {\element{solution}} element, where {\element{proof}} was allowed, we have
to allow {\element{proofobject}} as well. 

\subsection*{3.5 Adding Presentation Information to \ifpdf{OMDoc}\else{\omdoc}\fi}
\begin{enumerate}
\item It should be made clear that the {\attribute{xml:lang}{use,xslt,style}} attribute of the
  {\element{use}}, {\element{xslt}} and {\element{style}} elements does not have the default
  value {\ttin{en}}.
\item {\omdoc}1.1 uses the {\attribute{style}{*}} attribute for all elements that
  have an {\attribute{id}{*}} attribute to specify generic style classes for the
  {\omdoc} elements.  This is based on a misunderstanding of the {\xml}
  {\indextoo{cascading style sheet}} ({\css}) mechanism~\cite{BosHak:css98}, which
  uses the {\attribute{class}{*}} attribute to specify this information and uses the
  {\attribute{style}{*}} attribute to specify {\css} directives that override the class
  information. 
  
  {\sf Even though this is a grave error (it severely limits the usefulness) we
    will not change it in the {\omdoc} 1.1 specification and wait for the release
    of {\omdoc} 1.2 to fix this, the renaming of the {\attribute{style}{*}}
    attribute to {\attribute{class}{*}} would break existing implementations.}
\end{enumerate}

\subsubsection{3.5.2 Specifying the Notation of Mathematical Symbols}
\begin{description}
\item[Figure 3.44] the example still uses the {\omdoc}1.0 version of specifying
  {\xslt} content via the {\attribute{system}{use}} attribute, in {\omdoc}1.1 the
  element {\element{xslt}} should be used. 
\end{description}


\subsection*{3.6 Identifying and Referencing {\ifpdf{OMDoc}\else{\omdoc}\fi} Elements}

\subsection*{Locating {\ifpdf{OMS}\else{\element{OMS}}\fi} elements by the {\ifpdf{OMDoc}\else{\omdoc}\fi} Catalogue}
No errata known.

\subsection*{A URI-based Mechanism for Element Reference}
The text does not make it clear that the namespace prefixes for theory collections
can be declared in any element that dominates the referencing element. The DTD
does not allow this either. We will not change this in the DTD, since the changes
are too disruptive and {\omdoc}1.2 is coming up soon. 

\subsection*{Uniqueness Constraints and Relative URI references}
No errata known.
\section*{4 OMDoc Applications, Tools,  and Projects}
No errata known.

\section*{B Changes}
  No errata known.
\section*{C Quick-Reference for OMDoc Elements}
No errata known.
\section*{D Quick-Reference for OMDoc Attributes}
No errata known.
\section*{E OMDoc DTD}
We use the following {\indextoo{public identifier}} for DTDs: {\ttin{-//OMDoc//DTD OMDoc V1.1//EN}}
\begin{enumerate}
\item The {\attribute{xml:lang}{use, xslt, style}} attribute of the {\element{use}},
  {\element{xslt}} and {\element{style}} elements should not have the default value
  {\ttin{en}}.
\item The elements {\element{theory-inclusion}}, {\element{axiom-inclusion}} and
  {\element{decomposition}} may not occur in a {\element{theory}} element.
\item The content model for {\element{solution}} should make the {\element{FMP}},
  {\element{proof}}, and {\element{proofobjects}} elements optional.
\item the attribute {\attribute{for}{exercise}} should have been optional
\item the optional {\element{recognizer}} element should be a child of {\element{sortdef}},
  and not of the {\element{constructor}} element.
\item the {\element{exercise}} element should allow multiple mathematical objects,
  not at most one.
\item the {\ttin{\%cfm;}} parameter entity in the DTD did not allow for multiple
  FMPs, even though the spefication says that they appear in multi-logic groups.
\item the default {\attribute{precedence}{presentation}} attribute of the
  {\element{presentation}} element should have the default value 1000. 
\item the content model of the {\element{definition}} element had to be adapted to
  the clarification in the specification. The old version only allowed
  {\element{CMP}}-only content with {\tt{rxp}}. The value
  {\attval{obj}{type}{definition}} has also been dropped.
\item the content model for the {\element{omdoc}} element prescribed at least one
  omdoc item. This is not intended, since we want to allow catalogue-only
  documents for administrative purposes. 
\item the theory attribute for the {\element{assertion}} {\element{alernative}},
  {\element{proof}} and {\element{proofobject}} elements should be of type
  {\tt{CDATA}}, after all it contains a URI that points to an external theory.
\item the specification mentions type {\attval{comment}{type}{omtext}}, but the
  DTD did not allow this.
\end{enumerate}

%%% Local Variables: 
%%% mode: latex
%%% TeX-master: "errata"
%%% End: 
