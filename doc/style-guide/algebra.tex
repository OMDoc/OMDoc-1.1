\chapter{Formalizing Mathematics}
\label{cha:algebra}

In this chapter we will work an example of a stepwise formalization of
mathematical knowledge. This is the task of e.g. an editor of a mathematical
textbook preparing it for web-based publication. 

Since {\omdoc} only gives an infrastructure for modeling mathematical content and
not for the conceptualization of the mathematical objects themselves, we will
start from an informal, but rigorous text: a fragment of Bourbaki's
Algebra~\cite{Bourbaki:a74}, which we display in {\myfigref{bourbaki}} for
convenience.

We will now formalize this piece of mathematical knowledge in three stages in the
next sections, discussing the design decisions as we go along.

In section~\ref{sec:algebra1} we discuss the first level of formalization. At this
level we just mark up the large-scale structure of the text (which we assume to be
given as {\LaTeX} source code for simplicity) into {\omdoc} elements. At this
level, we have only marked up the text structure and mathematical statements. This
allows us to annotate and extract some metadata and would allow applications to
slice the text into individual units, store it in data bases like {\mbase},
or\ednote{slicing books}, or assemble the text slices into individualized books.
e.g. covering only a sub-topic of the orignial work.

However, all of the text itself, still contains the {\LaTeX} markup for formulae,
which is readable only by experienced humans, and can in particular only be
processed by


In section~\ref{sec:algebra2}, we will map all mathematical objects in the text
into {\openmath} formulae,\ednote{say something about the benefits} 

Finally, in section~\ref{sec:algebra3} we will fully formalize the mathematical
knowledge.  \ednote{say something about the benefits}



\begin{myfig}{bourbaki}{A fragment of Bourbaki's Algebra}
\fbox{\small
\begin{minipage}{12cm}\parindent=1.5em
\noindent{\bf 1. LAWS OF COMPOSITION}\vspace{1em}\par\noindent
{\sc Definition 1.} {\em Let $E$ be a set. A mapping of $E\times E$ is called a
  law of composition on $E$. The value $f(x,y)$ of $f$ for an ordered pair
  $(x,y)\in E\times E$ is called the composition of $x$ and $y$ under this law. A
  set with a law of composition is called a magma.}
\vspace{1em}

The composition of $x$ and $y$ is usually denoted by writing $x$ and $y$ in a
definite order and separating them by a characteristic symbol of the law in
question (a symbol which it may be agreed to omit). Among the symbols most often
used are + and ., the usual convention being to omit the latter if desired; with
these symbols the composition of $x$ and $y$ is written respectively as $x+y$,
$x.y$ or $xy$. A law denoted by the symbol $+$ is usually called {\em addition}
(the composition $x+y$ beind called the {\em sum} of $x$ and $y$) and we say that
it is {\em written additively}; a law denoted by the symbol . is usually called
{\em multiplication} (the composition $x.y=xy$ being called the {\em product} for
$x$ and $y$) and we say that it is {\em written multiplicatively}. 

In the general arguments of paragraphs 1 to 3 of this chapter we shall generally use the
symbols $\top$ and $\bot$ to denote arbitrary laws of composition.

By an abuse of language, a mapping of a {\em subset} of $E\times E$ into $E$ is sometimes
called a law of composition {\em not everywhere defined} on $E$.
\vspace{1em}

\strut\hfill
\begin{minipage}{11cm}\parindent=1.5em
  {\em Examples.} (1) The mappings $(X,Y)\mapsto X\cup Y$ and $(X,Y)\mapsto X\cap Y$ are
  laws of composition on the set of subsets of a set $E$.
 
  (2) On the set $\bf N$ of natural numbers, addition, multiplication, and exponentiation
  are laws of composition (the compositions of $x\in{\bf N}$ and $y\in{\bf N}$ under these
  laws being denoted respectively by $x+y$, $xy$, or $x.y$ and $x^y$)  ({\em Set Theory},
  III, $\P3$, no. 4).\ednote{right paragraph sign here?}
  
  (3) Let $E$ be a set; the mapping $(X,Y)\mapsto X\circ Y$ is a law of composition on the
  set of subsets of $E\times E$ ({\em Set Theory}, II, $\P3$, no. 3, Definition 6); the
  mapping $(f,g)\mapsto f\circ g$ is a law of composition on the set of mappings from 
  $E$ into $E$ ({\em Set Theory}, II, $\P5$, no. 2).
\end{minipage}
\end{minipage}}
\end{myfig}


\section{Marking up the Top-Level Structure and Mathematical Statements}\label{sec:algebra1}

In this section we conver the mathematical text in {\myfigref{bourbaki}} in a very
simple-minded manner. At this level, we only care about the top-level structure
and the mathematical statements. We will explain the features in detail by line
numbers below.  

{\scriptsize\listinginput[5]{1}{algebra1.omdoc}}
\begin{itemize}
\item[1] The {\xml} declaration. 
\item[2] The document type declaration. 
\item[4] the root element of the document, it must be {\element{omdoc}} and it must have
  an {\attribute{id}{omdoc}} attribute. Its value can be any string, but should not contain
  ``:'' (this is reserved for namespaces in {\xml}).\ednote{continue}
\end{itemize}

\section{Marking up the Formulae}\label{sec:algebra2}
Here is the converted text. We will explain the features in detail by line numbers  below.
{\scriptsize\listinginput[5]{1}{algebra2.omdoc}}
\begin{itemize}
\item[1] The {\xml} declaration. 
\item[2] The document type declaration. 
\item[4] the root element of the document, it must be {\element{omdoc}} and it must have
  an {\attribute{id}{omdoc}} attribute. Its value can be any string, but should not contain
  ``:'' (this is reserved for namespaces in {\xml}).\ednote{continue}
\end{itemize}


%%% Local Variables: 
%%% mode: latex
%%% TeX-master: "guide"
%%% End: 
