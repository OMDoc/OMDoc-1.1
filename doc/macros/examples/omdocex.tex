\begin{omdoc}

\begin{ommetadata} 
  \dccontributor{Contributor 1}
  \dccreator[edt]{Creator as the editor}
  \dctitle{The title of the document}
 here we define the metadata of the document
\end{ommetadata}

\begin{omomtext}{intro}
  \omrsrelation{introduction}{}{}
  \begin{omCMPverb}{omtext}
      We will start out with an introductory text
  \end{omCMPverb}
\end{omomtext}

\begin{omtheory}{testtheory}
\omcommonname[deu]{Monoid Theorie}

\begin{omsymbol}{monoid}
  \begin{omCMPverb}{omtext}
    The monoids that we all know and love
  \end{omCMPverb}
  \begin{omCMPverb}[deu]{omtext}
    Die Monoide, wie wir sie alle kennen und lieben
  \end{omCMPverb}
  \omcommonname[deu]{plus}
  \omcommonname{plus}
  \begin{omsignatureverb}{POST}
    <OMOBJ><OMA><OMS cd="mltt" name="funtype"/><OMV name="alpha"/></OMA></OMOBJ>
  \end{omsignatureverb}
  here we define monoids 
\end{omsymbol}


\begin{omomtext}{testremark}
  \omrsrelation{introduction}{}{}
  \begin{omCMP}{omtext}
    \begin{omverb}then we will use an omnote to
    \end{omverb} 
    \begin{omomletverb}{appl1}{js}{sdlfkj}{call-mint}
       explain 
    \end{omomletverb}
    \begin{omverb}
      the symbol
    \end{omverb}
  \end{omCMP}
\end{omomtext}

\begin{omdefinition}{monoiddef}{monoid}
  \begin{omCMPverb}{omtext}
    Here comes the definition of  monoids
  \end{omCMPverb}
  \begin{omFMP}
    \((#test:sym:$X:$Y:)\)
  \end{omFMP}
  sdfsdfsdf
\end{omdefinition}


\begin{omomtext}{testlinkage}
  \omrsrelation{linkage}{monoiddef}{recursivemonoid}
  \begin{omCMPverb}{omtext}
    then a omlinkage to link the previous definition to the next
  \end{omCMPverb}
\end{omomtext}

\begin{omdefinition}{recursivemonoid}{monoid}
  \begin{omCMPverb}{omtext}
    Here comes a recursive definition
  \end{omCMPverb}
  \begin{omrequation}
    \<(#arith1:plus:$X:#arith1:zero:)\>
  \goesto
    \($X:\)
  \end{omrequation}
  sdfsdfsdf
\end{omdefinition}

\end{omtheory}

\begin{omassertion}{testtheorem}{testtheory}{theorem}
  \begin{omCMPverb}{omtext}
    A theorem
  \end{omCMPverb}
  \begin{omFMP}
    \({#quant1:forall:$X:$Y:.(#relation:eq:$X:$Y:)}\)
  \end{omFMP}
\end{omassertion}

\begin{omassertion}{testlemma}{testtheory}{lemma}
  \begin{omCMPverb}{omtext}
    Transitivity of equality
  \end{omCMPverb}
  \begin{omassumption}{A1}
    \begin{omCMPverb}{omtext}
    \($X:\) and  \($Y:\) are equal
    \end{omCMPverb}
    \begin{omFMP}
    \((#relation:eq:$X:$Y:)\)
    \end{omFMP}
  \end{omassumption}
  \begin{omassumption}{A2}
    \begin{omCMPverb}{omtext}
    \($Y:\) and  \($Z:\) are equal
    \end{omCMPverb}
    \begin{omFMP}
    \((#relation:eq:$Y:$Z:)\)
    \end{omFMP}
  \end{omassumption}
  \begin{omconclusion}{C1}
    \begin{omCMPverb}{omtext}
      All objects are equal
    \end{omCMPverb}
    \begin{omFMP}
    \({#quant1:forall:$X:$Y:.(#relation:eq:$X:$Y:)}\)
    \end{omFMP}
  \end{omconclusion}
\end{omassertion}


\begin{omproof}{P1}{testlemma}{testtheory}
  \begin{omCMPverb}{omtext}
    A proof with two proof steps  for the lemma above
  \end{omCMPverb}
  \begin{omderive}{D1}
    \begin{omCMPverb}{omtext}
      The first proof step consis
    \end{omCMPverb}
    \begin{omassumption}{D1.A}
      \begin{omCMPverb}
        This proof step makes a local hypothesis
      \end{omCMPverb}
    \end{omassumption}
    \begin{omconclusion}{D1.C}
    \begin{omCMPverb}{omtext}
      The assertion of the first proof step
    \end{omCMPverb}
    \begin{omFMP}
      \(#test:formula:\)
    \end{omFMP}
  \end{omconclusion}
\end{omderive}
  \begin{omderive}{D2}
    \begin{omCMPverb}{omtext}
      The second step
    \end{omCMPverb}
    \begin{omconclusion}{D2.c}
    \begin{omFMP} 
      \(#test:formula:\)
    \end{omFMP}
  \end{omconclusion}
\end{omderive}
  \begin{omconclude}{Conc}
    \begin{omCMPverb}{omtext}
      The last proof step
    \end{omCMPverb}
\end{omconclude}
\end{omproof}
\end{omdoc}

%%% Local Variables: 
%%% mode: latex
%%% TeX-master: "latex2omdoc"
%%% End: 
