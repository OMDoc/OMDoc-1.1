\documentclass{article}
\usepackage{../latex2omdoc}
\usepackage{ed}
\usepackage{url,moreverb,a4wide,array}
\def\xml{{\sc XmL}}
\def\xsl{{\sc XSL}}
\def\openmath{{\sc OpenMath}}
\def\omdocname{{\sc OMDoc}}
\dtdurl{../../../dtd/omdoc.dtd}
\begin{document}
\title{Creating {\sc OMDoc} representations from {\LaTeX}}
\author{Michael Kohlhase, Universit\"at des Saarlandes, Germany\\
  \url{http://www.ags.uni-sb.de/~kohlhase}}
\maketitle
\begin{abstract}
  In this paper we describe the {\LaTeX} style {\url{latex2omdoc.sty}} that allows
  to create {\openmath} objects and {\omdocname} representations from inside
  {\LaTeX} documents.
  
  This is useful when migrating existing mathematical texts from {\LaTeX}
  representation, e.g. in papers to {\omdocname}, or as a maintainance tool for
  authors that are more efficient/used to generating {\LaTeX} than {\xml}.
\end{abstract}

\section{Introduction}
{\omdocname}~\cite{Kohlhase:otormd99} is an {\xml}~\cite{bray:XML97} representation
format for mathematics based on {\openmath}~\cite{CapCoh:doms98}.  Since it is
tedious to write the {\xml} and the {\openmath} objects in the original form, we
provide {\LaTeX} macros that allow to specify it inside {\LaTeX} documents.



\section{{\openmath} Objects}\label{sec:openmath}
In this section, we will first briefly review the possible types of {\openmath}
objects and show the two {\LaTeX} representations provided by the
{\url{latex2omdoc}} style (full {\LaTeX} mode and abbreviation mode; see
Figure~\ref{fig:openmath}).

Both modes write the generated {\openmath} representations to an auxilary file
that can be specified by the user in the {\tt omoutput} environment.
{\verb+\begin{omoutput}[test.xml]+} will prepare a file {\url{test.xml}} for
  output and direct all {\openmath} output to it. The default for the optional
  argument is the name of the top-level {\LaTeX} file with extension {\tt xml},
  {\verb+\end{omoutput}+} will close the file, so that it can for instance
directly be inserted into the {\LaTeX} itself, say with a {\verb+\verbatiminput+}
statement from the {\url{moreverb.sty}} style (see also the source of this
document in section~\ref{sec:useful-macs}). Note that the second call to an
{\tt omoutput} environment with the same file name will overwrite a first one.

The {\tt omoutput} environment has a variant {\tt OMoutput}, which wraps the
output with an {\xml} declaration, and declaration of the document type definition
so that the result can be used as a standalone {\xml} file that can e.g. be
validated. For this it is necessary to specify the specify the URL of the {\xml}
document type definition (DTD) with the {\verb+\dtdurl+} command, the default
setting is {\tt omdoc.dtd}, i.e. assuming that the dtd is in the same directory as
the top-level {\LaTeX} file. For instance
{\verb+\dtdurl{http://www.mathweb.org/dtd/omdoc.dtd}+} will use the current
distribution version.

Note that {\tt omoutput} and {\tt OMoutput} environments can be nested (both
individually and among each other up to 15 levels), so that it is possible to
generate more than one file at a time. Since the DTD will probably not vary
between these, it is probably good to specify {\verb+\dtdurl+} in the document
header.

The abbreviation mode can be turned on by the environment given by the bracketings
{\verb+\<+} and {\verb+\>+} or {\verb+\(+} and {\verb+\)+}. The latter differs
from the former in that it additionally wraps an {\verb+<OMDOC>+} element
bracketing in the output (this is often a useful abbreviation).

Figure~\ref{fig:omabstr} shows an example of a $\lambda$-abstraction
$\lambda X Y.X^{-Y}$ and figure~\ref{fig:omattr} an attribuition of the variable $X$
with the type $(\iota\rightarrow o)$ and the color green.

\begin{figure}[htbp]
  \begin{center}
\begin{tabular}{|p{2cm}|l|l|l|}\hline
 category & 
  {\xml} representation & 
  full {\LaTeX} & 
  abbreviation \\\hline\hline
 symbol   &  
  {\small{\verb+<OMS cd="CD" name="N"/>+}} & 
  {\verb+\oms{CD}{N}+} &
  {\verb+#CD:N:+}\\\hline
 variable & 
  {\small{\verb+<OMV name="N"/>+}} & 
  {\verb+\omv{N}+} & 
  {\verb+$N:+} \\\hline
 application & 
 \begin{minipage}{3.5cm}
\begin{verbatim}
<OMA>
  <OM*>...<OM*>
</OMA>
\end{verbatim}
 \end{minipage} &
 \begin{minipage}{3.5cm}
\begin{verbatim}
\begin{oma}
  <OM*>...<OM*>
\end{oma}
\end{verbatim}
 \end{minipage} &
 \begin{minipage}{2cm}
\begin{verbatim}
(<OM*>
 ...<OM*>)
\end{verbatim}
 \end{minipage} \\\hline
 binding structure &
  \begin{minipage}{3.5cm}\small
\begin{verbatim}
<OMBIND>
  <OMS.../>
  <OMBVAR>
    <OMV*>...<OMV*>
  </OMBVAR>
  <OM*>
</OMBIND>
\end{verbatim}
  \end{minipage} & 
  \begin{minipage}{3.5cm}\small
\begin{verbatim}
\begin{ombind}
   \oms{a}{b}
   \begin{ombvar}
     \omv{x}...\omv{z}
   \end{ombvar}
   <OM*>
\end{ombind}
\end{verbatim}
  \end{minipage} &
  \begin{minipage}{2cm}
\begin{verbatim}
{#a:b:
 $x:$z:
.<OM*>}
\end{verbatim}
  \end{minipage}\\\hline
 attribution &
  \begin{minipage}{3.5cm}\small
\begin{verbatim}
<OMATTR>
   <OMATP>
     <OMS*><OM*>...
     <OMS*><OM*>
   </OMATP>
   <OM*>
</OMATTR>
\end{verbatim}
  \end{minipage} &
  \begin{minipage}{3.5cm}\small
\begin{verbatim}
\begin{omattr}
  \begin{omatp}
    \oms{a}{b}<OM*>
    \oms{c}{d}<OM*>
   \end{omatp}
   <OM*>
\end{omattr}
\end{verbatim}
  \end{minipage} &
  \begin{minipage}{2cm}\small
\begin{verbatim}
[#a:b:<OM*>,
 #c:d:<OM*>
|<OM*>]
\end{verbatim}
  \end{minipage}\\\hline\hline
\multicolumn{4}{|p{12cm}|}{{\tt <OM*>} stands for an arbitrary {\openmath} 
  object and {\tt <OMV*>} stands for a variable or an attribuited variable.}\\\hline
\end{tabular}
    \caption{{\LaTeX} representations of {\openmath} objects}
    \label{fig:openmath}
  \end{center}
\end{figure}
\begin{figure}[htbp]
  \begin{center}
    \begin{tabular}{|l|l|}\hline
\begin{minipage}{3.5cm}\small
\begin{verbatim}
\begin{omobj}
  \begin{ombind}
    \omsref{lambda}
    \begin{ombvar}
      \omv{X}
      \omv{Y}
    \end{ombvar}
    \begin{oma}
      \omsref{exp}
      \omv{X}
      \begin{oma}
        \omsref{inv}
        \omv{Y}
      \end{oma}
    \end{oma}
  \end{ombind}
\end{omobj}
\end{verbatim}
\end{minipage}&
\begin{minipage}{8cm}\small
\begin{verbatim}
<OMOBJ>
  <OMBIND>
    <OMS cd="ecc" name="lambda"/>
    <OMBVAR>
      <OMV name="X"/>
      <OMV name="Y"/>
    </OMBVAR>
    <OMA>
      <OMS cd="arith1" name="exp"/>
      <OMV name="X"/>
      <OMA>
        <OMS cd="arith1" name="unary-minus"/>
        <OMV name="Y"/>
      </OMA>
    </OMA>
  </OMBIND>
</OMOBJ>
\end{verbatim}
\end{minipage}\\\hline
 $\lambda X Y.X^{-Y}$ & 
\begin{minipage}{9cm}
\begin{verbatim}
\({#ecc:lambda:$X:(#arith1:unary-minus:$Y:)}\)
\end{verbatim}
\end{minipage}\\\hline
  \end{tabular}
    \caption{The representations of the functions $\lambda X Y.X^{-Y}$}
    \label{fig:omabstr}
  \end{center}
\end{figure}

\begin{figure}[htbp]
  \begin{center}
    \begin{tabular}{|l|l|}\hline
\begin{minipage}{5.5cm}
\begin{verbatim}
\begin{omattr}
  \begin{omatp}
    \oms{ecc}{type}
    \begin{oma}
      \oms{mltt}{funtype}
      \oms{typeind}
      \oms{typebool}
    \end{oma}
    \oms{colors}{colorattr}
    \oms{colors}{green}
  \end{omatp}
\end{omattr}
\end{verbatim}
\end{minipage} &
\begin{minipage}{8cm}\small
\begin{verbatim}
<OMATTR>
  <OMATP>
    <OMS cd="ecc" name="type"/>
    <OMA>
      <OMS cd="mltt" name="funtype"/>
      <OMS cd="types" name="individuals"/>
      <OMS cd="types" name="booleans"/>
    </OMA>  
    <OMS cd="colors" name="colorattr"/>
    <OMS cd="colors" name="green"/>
  </OMATP>
  <OMV name="X"/>
</OMATTR>
\end{verbatim}
\end{minipage}\\\hline
$X_{\iota\rightarrow o}^{green}$&
\begin{minipage}{8cm}
\begin{verbatim}
\<[#ecc:type:,
    (#mltt:funtype:#types:ind:#types:bool:);
   #colors:colorattr:,#colors:green:
  |$X:]\>
\end{verbatim}
\end{minipage}\\\hline
\end{tabular}
\caption{An {\openmath} attribuition}
    \label{fig:omattr}
  \end{center}
\end{figure}

\begin{figure}[htbp]
  \begin{center}
{\footnotesize\begin{verbatim}
<OMA>                                  <OMA>
  <OMV name="f"/>                        <OMV name="f"/>
  <OMA id="a">                           <OMA>
    <OMV name="g"/>                        <OMV name="g"/>
    <OMV name="x" id="aa"/>                 <OMV name="x"/>
    <OMV xref="2"/>                        <OMV name="x"/>
  </OMA>                                 </OMA>
  <OMA xref="first"/>                    <OMA>
</OMA>                                     <OMV name="g"/>
                                           <OMV name="x"/>
                                           <OMV name="x"/>
\(($f:~a($g:~<aa>$x:^<aa>$:^a())\)       </OMA>
                                       </OMA>
\end{verbatim}
}
    \caption{Reusing {\openmath} objects by reference}
    \label{fig:reference}
  \end{center}
\end{figure}
%%% Local Variables: 
%%% mode: latex
%%% TeX-master: "latex2omdoc"
%%% End: 

Note that these macros do not check for validity of the {\xml} output, this is
left to an {\xml} validator. Moreover, little work has gone into generating good
error messages. In particular, some of the macros in the abbreviated mode make
assumptions about the {\openmath}-well-formedness; most notably, the start tag
{\verb+{+} for an {\openmath} binding object only works, iff the next object is a
valid {\openmath} symbol, i.e. of the form {\verb+#cd:name:+}.

{\tt latex2omdoc.sty} also supports {\omdocname}'s {\tt id}/{\tt xref} extension
to {\openmath} objects. In short, this allows to re-use {\openmath} objects by
reference. For instance, the left hand side {\omdocname} element in
figure~\ref{fig:reference} can be realized by the expresssion below it and has the
same meaning as the {\openmath} object on the right hand side. {\tt
  latex2omdoc.sty} provides two primitives for this: {\verb+~+} for adding {\tt
  id} attributes to the next element. This takes one token as arguemnt, so if the
attribute needs more than one letter, it has to be grouped by {\tt <} and {\tt >}.
{\verb+^+} for adding {\tt xref} attributes to the next element, the same token
conventions as above apply.

\subsection{Defining Macros}\label{sec:macros}

One of the primary advantages of using {\LaTeX} to generate {\omdocname} documents
over directly writing them in a text editor is that {\LaTeX} allows to define
macros. They can be simply be defined using {\LaTeX}'s {\verb+\(re)newcommand+},
{\verb+\(re)newenvironment+} facilities. Currenly, {\url{latex2omdoc.sty}} only
allws to use the full-form macros in definitions. For instance, we can use the
definitions to simplify the attribuition example above to
{\verb+\<\oftypeiogreen<$X:>\>+}.

{\footnotesize\begin{verbatim}
\newcommand{\typeio}%             the type of unary predicates on individuals
{\begin{oma}\oms{mltt}{funtype}\oms{types}{individuals}\oms{types}{booleans}\end{oma}}

\newcommand{\colgreen}%           the attribute value pair for the color green
{\oms{colors}{colorattr}\oms{colors}{green}}

\newcommand{oftypeiogreen}[1]%    attribuition stating that the content is of \typeio
{\begin{omattr}\begin{omatp}\oms{ecc}{type}\typio\colgreen\end{omatp}#1\end{omattr}}
\end{verbatim}
}

\section{{\omdocname} Elements}\label{sec:omdoc}
In figure~\ref{fig:elements}, we have given an overview over the macros that write
{\omdocname} material for the {\xml} file. Since these generate {\omdocname}
output, they must be enclosed in an {\tt omdoc} environment. This is a variant of
the {\tt OMoutput} environment that also wraps the content into an {\tt <omdoc>}
element that is the {\xml} top-level element for {\omdocname} documents. 

Note that as in the case of the {\openmath} macros described in
section~\ref{sec:openmath}, they do not check for validity of the {\xml} output,
this is left to an {\xml} validator.



\begin{figure}[htbp]
\begin{center}
\begin{tabular}{|>{\tt}l|l|l|l|}\hline
{\rm macro}   & Type   & Arguments           & Env \\\hline\hline
 omdoc        & root   & label               & + \\\hline
 omgroup      & struc  & label, type         & + \\\hline
 omtext       & text   & label               & + \\\hline
 omCMP        & text   & [lang], label       & + \\\hline
 omrsrelation & text   & type, for, from     & + \\\hline\hline
 omassertion  & math   & label, theory, type & + \\\hline
 omFMP        & math   &                     & + \\\hline
 omassumption & math   & label               & + \\\hline
 omconclusion & math   & label               & + \\\hline
 omproof      & math   & label, for, theory  & + \\\hline
 omexample    & math   & label, for          & + \\\hline\hline
 omhypothesis & proof  & label               & + \\\hline
 omderive     & proof  & label               & + \\\hline
 omconclude   & proof  & label               & + \\\hline
 ommethod     & proof  & theory, name        & - \\\hline
 omparameter  & proof  & ----                & + \\\hline
 ompremise    & proof  & refersto            & - \\\hline
 omextpremise & proof  & href                & - \\\hline\hline
 omsymbol     & theory & label               & + \\\hline
 omstructure  & theory & label               & + \\\hline
 omfeature    & theory & label               & + \\\hline
 omcommonname & theory & [lang]              & + \\\hline
 omdefinition & theory & label, for          & + \\\hline\hline
 omexercise   & aux    & label, for          & + \\
 omsolution   & aux    & label               & + \\\hline
 omhint       & aux    & label               & + \\\hline
 ommcgroup    & aux    & ----                & + \\\hline
 ommc         & aux    & label               & + \\\hline
 omchoice     & aux    & ----                & + \\\hline
 omanswer     & aux    & verdict             & + \\\hline
 omomlet      & aux    & href                & + \\\hline
 omref        & aux    & tolabel             & - \\\hline
\end{tabular}
    \caption{\protect{\omdocname} elements}
    \label{fig:elements}
  \end{center}
\end{figure}

\section{Useful Redefinitions of commonly used Macros}\label{sec:useful-macs}

The {\url{latex2omdoc.sty}} style file can be used as a migration tool from
{\LaTeX} texts to {\omdocname} documents. In this section, we will present a set
of useful macros that can be used transport much of the structure that is present
in a well-structured mathematical {\LaTeX} document (definition, theorem, proof)
into {\omdocname}.\ednote{continue}
\begin{figure}[htbp]
  \begin{center}
    \fbox{\begin{minipage}{12cm}\scriptsize\verbatiminput{migrationex}\end{minipage}}
    \caption{The file {\tt migrationex.tex}}
    \label{fig:migrationex.tex}
  \end{center}
\end{figure}

For instance there are the environments {\tt theorem}, {\tt lemma}, {\tt
  corollay}, {\tt conjecture}, {\tt definition}, {\tt remark} that allwo to use
the file {\url{migrationex.tex}}, in figure~\ref{fig:migrationex.tex} to generate
the {\omdocname} file {\url{migration.xml}} in figure~\ref{fig:migrationex.xml},
generating the DVI output in figure~\ref{fig:migrationex.dvi} in the process.
\begin{figure}[htbp]
  \begin{center}
    \fbox{\begin{minipage}{12cm}\scriptsize\newtheorem{thm}{Theorem}[section]
\newtheorem{defn}{Definition}[thm]
\begin{defn}[Monoid]
    A monoid is a semigroup with a unit element
\end{defn}
\begin{thm}[Hauptsatz]
    A monoid has at most one unit.
\end{thm}

%%%%%%% migrates to 
%%%%%%% (we leave the stuff up there to see the old version in the DVI)

\begin{omdoc}[migrationex.xml]
\ommiglanguage{eng}
\begin{ommetadata}
\dctitle{Monoids migrated from LaTeX}
\end{ommetadata}

\begin{omtheory}{monoids}
\ommigtheory{monoids}
\begin{definition}[Monoid]
  \begin{omverb}
    A monoid is a semigroup with a unit element
  \end{omverb}
\end{definition}
\end{omtheory}

\begin{remark}[Hauptsatz]{monoids}
  \begin{omverb}
    A monoid has at most one unit.
  \end{omverb}
\end{remark}

\begin{omassertion}{sin}{trig}{theorem}
  \begin{omFMP}
    \mathematica{sin[1+1] == 2}
  \end{omFMP}
\end{omassertion}

\end{omdoc}
%%% Local Variables: 
%%% mode: latex
%%% TeX-master: "latex2omdoc"
%%% End: 
\end{minipage}}
    \caption{The DVI output of {\tt migrationex.tex}}
    \label{fig:migrationex.dvi}
  \end{center}
\end{figure}
\begin{figure}[htbp]
  \begin{center}
    \fbox{\begin{minipage}{12cm}\scriptsize\verbatiminput{migrationex.xml}\end{minipage}}
    \caption{The file {\tt migrationex.xml} generated form {\tt migrationex.tex}}
    \label{fig:migrationex.xml}
  \end{center}
\end{figure}
As we can see for migrating the a file with {\LaTeX} markup like the first part
(above the comment) of {\url{migrationex.tex}} in
figure~\ref{fig:migrationex.tex}, can be migrated by using the corresponding
migration macros provided by {\url{latex2omdoc.sty}}, and adding the {\tt omverb }
environment for all parts that should go to the {\xml} file verbatim.

Of course we have to specify the language used in the texts in the document using
a {\verb+\ommiglanguage+} statment and the {\omdocname} theory by
{\verb+\ommigtheory+}. These two details go into the {\xml} file and cannot be
deduced by {\url{latex2omdoc.sty}} itself.


\section{Conclusion and Wishlist}\label{sec:concl}

We have presented a {\LaTeX} style file that allows to generate {\omdocname}
representations from {\LaTeX} files. This gives us a productivity tool for users
that are accustomed to generating {\LaTeX}, or have large legacy {\LaTeX}
representations of mathematical knowledge.

This style file is far from finished, so please help it, if you can. The wishlist
for {\url{latex2omdoc.sty}} includes the following items
\begin{enumerate}
\item Indentation in the {\xml} file, to make the output more readable. 
\item a definition mechanism that uses the abbreviated form for {\openmath}
  objects.
\item better error messages, and some validity warnings. 
\item extend the abbreviated mode by parser tags for {\tt <OMSTR>}, {\tt <OMI>},
  {\tt <OMF>},\ldots
\item an {\xsl} style sheet to generate {\tt latex2omdoc} input from {\omdocname}
  representations.
\end{enumerate}

\bibliographystyle{alpha}
\bibliography{miko}

\newpage
\begin{appendix}
\section{A {\ttin{latex2omdoc}} Example}\label{sec:omdocex}
In this section, we will look at an example of an {\omdocname} file generated by
{\url{latex2omdoc.sty}} 
\subsection{The \protect{\LaTeX} source file}
{\scriptsize\verbatiminput{omdocex}}\newpage
\subsection{The log information in the {\sc dvi} file}
\begin{omdoc}

\begin{ommetadata} 
  \dccontributor{Contributor 1}
  \dccreator[edt]{Creator as the editor}
  \dctitle{The title of the document}
 here we define the metadata of the document
\end{ommetadata}

\begin{omomtext}{intro}
  \omrsrelation{introduction}{}{}
  \begin{omCMPverb}{omtext}
      We will start out with an introductory text
  \end{omCMPverb}
\end{omomtext}

\begin{omtheory}{testtheory}
\omcommonname[deu]{Monoid Theorie}

\begin{omsymbol}{monoid}
  \begin{omCMPverb}{omtext}
    The monoids that we all know and love
  \end{omCMPverb}
  \begin{omCMPverb}[deu]{omtext}
    Die Monoide, wie wir sie alle kennen und lieben
  \end{omCMPverb}
  \omcommonname[deu]{plus}
  \omcommonname{plus}
  \begin{omsignatureverb}{POST}
    <OMOBJ><OMA><OMS cd="mltt" name="funtype"/><OMV name="alpha"/></OMA></OMOBJ>
  \end{omsignatureverb}
  here we define monoids 
\end{omsymbol}


\begin{omomtext}{testremark}
  \omrsrelation{introduction}{}{}
  \begin{omCMP}{omtext}
    \begin{omverb}then we will use an omnote to
    \end{omverb} 
    \begin{omomletverb}{appl1}{js}{sdlfkj}{call-mint}
       explain 
    \end{omomletverb}
    \begin{omverb}
      the symbol
    \end{omverb}
  \end{omCMP}
\end{omomtext}

\begin{omdefinition}{monoiddef}{monoid}
  \begin{omCMPverb}{omtext}
    Here comes the definition of  monoids
  \end{omCMPverb}
  \begin{omFMP}
    \((#test:sym:$X:$Y:)\)
  \end{omFMP}
  sdfsdfsdf
\end{omdefinition}


\begin{omomtext}{testlinkage}
  \omrsrelation{linkage}{monoiddef}{recursivemonoid}
  \begin{omCMPverb}{omtext}
    then a omlinkage to link the previous definition to the next
  \end{omCMPverb}
\end{omomtext}

\begin{omdefinition}{recursivemonoid}{monoid}
  \begin{omCMPverb}{omtext}
    Here comes a recursive definition
  \end{omCMPverb}
  \begin{omrequation}
    \<(#arith1:plus:$X:#arith1:zero:)\>
  \goesto
    \($X:\)
  \end{omrequation}
  sdfsdfsdf
\end{omdefinition}

\end{omtheory}

\begin{omassertion}{testtheorem}{testtheory}{theorem}
  \begin{omCMPverb}{omtext}
    A theorem
  \end{omCMPverb}
  \begin{omFMP}
    \({#quant1:forall:$X:$Y:.(#relation:eq:$X:$Y:)}\)
  \end{omFMP}
\end{omassertion}

\begin{omassertion}{testlemma}{testtheory}{lemma}
  \begin{omCMPverb}{omtext}
    Transitivity of equality
  \end{omCMPverb}
  \begin{omassumption}{A1}
    \begin{omCMPverb}{omtext}
    \($X:\) and  \($Y:\) are equal
    \end{omCMPverb}
    \begin{omFMP}
    \((#relation:eq:$X:$Y:)\)
    \end{omFMP}
  \end{omassumption}
  \begin{omassumption}{A2}
    \begin{omCMPverb}{omtext}
    \($Y:\) and  \($Z:\) are equal
    \end{omCMPverb}
    \begin{omFMP}
    \((#relation:eq:$Y:$Z:)\)
    \end{omFMP}
  \end{omassumption}
  \begin{omconclusion}{C1}
    \begin{omCMPverb}{omtext}
      All objects are equal
    \end{omCMPverb}
    \begin{omFMP}
    \({#quant1:forall:$X:$Y:.(#relation:eq:$X:$Y:)}\)
    \end{omFMP}
  \end{omconclusion}
\end{omassertion}


\begin{omproof}{P1}{testlemma}{testtheory}
  \begin{omCMPverb}{omtext}
    A proof with two proof steps  for the lemma above
  \end{omCMPverb}
  \begin{omderive}{D1}
    \begin{omCMPverb}{omtext}
      The first proof step consis
    \end{omCMPverb}
    \begin{omassumption}{D1.A}
      \begin{omCMPverb}
        This proof step makes a local hypothesis
      \end{omCMPverb}
    \end{omassumption}
    \begin{omconclusion}{D1.C}
    \begin{omCMPverb}{omtext}
      The assertion of the first proof step
    \end{omCMPverb}
    \begin{omFMP}
      \(#test:formula:\)
    \end{omFMP}
  \end{omconclusion}
\end{omderive}
  \begin{omderive}{D2}
    \begin{omCMPverb}{omtext}
      The second step
    \end{omCMPverb}
    \begin{omconclusion}{D2.c}
    \begin{omFMP} 
      \(#test:formula:\)
    \end{omFMP}
  \end{omconclusion}
\end{omderive}
  \begin{omconclude}{Conc}
    \begin{omCMPverb}{omtext}
      The last proof step
    \end{omCMPverb}
\end{omconclude}
\end{omproof}
\end{omdoc}

%%% Local Variables: 
%%% mode: latex
%%% TeX-master: "latex2omdoc"
%%% End: 
\newpage
\subsection{The {\xml} document generated by this file}
{\scriptsize\verbatiminput{\jobname.xml}}
\end{appendix}
\end{document}

%%% Local Variables: 
%%% mode: latex
%%% TeX-master: t
%%% End: 
