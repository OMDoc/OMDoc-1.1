\documentclass{article}

\begin{document}
\noindent{\bf 1. LAWS OF COMPOSITION}\vspace{1em}\par\noindent
{\sc Definition 1.} {\em Let $E$ be a set. A mapping of $E\times E$ is called a
  law of composition on $E$. The value $f(x,y)$ of $f$ for an ordered pair
  $(x,y)\in E\times E$ is called the composition of $x$ and $y$ under this law. A
  set with a law of composition is called a magma.}  \vspace{1em}
      
The composition of $x$ and $y$ is usually denoted by writing $x$ and $y$ in a
definite order and separating them by a characteristic symbol of the law in
question (a symbol which it may be agreed to omit). Among the symbols most often
used are + and ., the usual convention being to omit the latter if desired; with
these symbols the composition of $x$ and $y$ is written respectively as $x+y$,
$x.y$ or $xy$. A law denoted by the symbol $+$ is usually called {\em addition}
(the composition $x+y$ beind called the {\em sum} of $x$ and $y$) and we say that
it is {\em written additively}; a law denoted by the symbol . is usually called
{\em multiplication} (the composition $x.y=xy$ being called the {\em product} for
$x$ and $y$) and we say that it is {\em written multiplicatively}.
      
In the general arguments of paragraphs 1 to 3 of this chapter we shall generally
use the symbols $\top$ and $\bot$ to denote arbitrary laws of composition.
      
By an abuse of language, a mapping of a {\em subset} of $E\times E$ into $E$ is
sometimes called a law of composition {\em not everywhere defined} on $E$.
\vspace{1em}
      
\strut\hfill
\begin{minipage}{11cm}\parindent=1.5em
  {\em Examples.} (1) The mappings $(X,Y)\mapsto X\cup Y$ and $(X,Y)\mapsto X\cap
  Y$ are laws of composition on the set of subsets of a set $E$.
  
  (2) On the set $\bf N$ of natural numbers, addition, multiplication, and
  exponentiation are laws of composition (the compositions of $x\in{\bf N}$ and
  $y\in{\bf N}$ under these laws being denoted respectively by $x+y$, $xy$, or
  $x.y$ and $x^y$) ({\em Set Theory}, III, $\P3$, no. 4).
  
  (3) Let $E$ be a set; the mapping $(X,Y)\mapsto X\circ Y$ is a law of
  composition on the set of subsets of $E\times E$ ({\em Set Theory}, II, $\P3$,
  no. 3, Definition 6); the mapping $(f,g)\mapsto f\circ g$ is a law of
  composition on the set of mappings from $E$ into $E$ ({\em Set Theory}, II,
  $\P5$, no. 2).
\end{minipage}

\end{document}
